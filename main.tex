
\documentclass[11pt,a4paper,oneside]{book}
\usepackage{a4wide}                     % Iets meer tekst op een bladzijde
\usepackage[dutch]{babel}               % Voor nederlandstalige hyphenatie (woordsplitsing)
\usepackage{amsmath}                    % Uitgebreide wiskundige mogelijkheden
\usepackage{amssymb}                    % Voor speciale symbolen zoals de verzameling Z, R...
\usepackage{makeidx}                    % Om een index te maken
\usepackage{url}                        % Om url's te verwerken
\usepackage{graphicx}                   % Om figuren te kunnen verwerken
\usepackage[small,bf,hang]{caption}     % Om de captions wat te verbeteren
\usepackage{xspace}                     % Magische spaties na een commando
\usepackage[utf8]{inputenc}           % Om niet ascii karakters rechtstreeks te kunnen typen
\usepackage{float}                      % Om nieuwe float environments aan te maken. Ook optie H!
\usepackage{flafter}                    % Opdat floats niet zouden voorsteken
\usepackage{listings}                   % Voor het weergeven van letterlijke text en codelistings
\usepackage[round]{natbib}              % Voor auteur-jaar citaties.
\usepackage[nottoc]{tocbibind}		% Bibliografie en inhoudsopgave in ToC; zie tocbibind.dvi
\usepackage{eurosym}                    % om het euro symbool te krijgen
\usepackage{textcomp}                   % Voor onder andere graden celsius
\usepackage{fancyhdr}                   % Voor fancy headers en footers
\usepackage[Gray,squaren,thinqspace,thinspace]{SIunits} % Om elegant eenheden te zetten
\usepackage[version=3]{mhchem}          % Voor elegante scheikundige formules

% Volgend package is niet echt nodig. Het laat echter toe om gemakkelijk elektronisch
% te navigeren in je pdf-document. Deze package moet altijd als laatste ingeladen worden.
\usepackage[a4paper,plainpages=false]{hyperref}    % Om hyperlinks te hebben in het pdfdocument.


%%%%%%%%%%%%%%%%%%%%%%%%%%%%%%
% Algemene instellingen van het document.
%%%%%%%%%%%%%%%%%%%%%%%%%%%%%%

% De splitsingsuitzonderingen
\hyphenation{back-slash split-sings-uit-zon-de-ring}

%\bibpunct{(}{)}{;}{y}{,}{,}             % Auteur-jaar citaties -- zie natbib.dvi voor meer uitleg; niet echt nodig

% Het bibliografisch opmaak bestand.
% ZORG ERVOOR DAT bibliodutch.bst ZICH IN JE WERKDIRECTORY BEVINDT!!!
\bibliographystyle{bibliodutch}

\setlength{\parindent}{0cm}             % Inspringen van eerste lijn van paragrafen is niet gewenst.

\renewcommand{\baselinestretch}{1.2} 	% De interlinie afstand wat vergroten.

\graphicspath{{figuren/}}               % De plaars waar latex zijn figuren gaat halen.

\makeindex                              % Om een index te genereren.

\setcounter{MaxMatrixCols}{20}          % Max 20 kolommen in een matrix

% De headers die verschijnen bovenaan de bladzijden, herdefinieren:
\pagestyle{fancy}                       % Om aan te geven welke bladzijde stijl we gebruiken.
\fancyhf{}                              % Resetten van al de fancy settings.
\renewcommand{\headrulewidth}{0pt}      % Geen lijn onder de header. Zet dit op 0.4pt voor een mooie lijn.
\fancyhf[HL]{\nouppercase{\textit{\leftmark}}} % Links in de header zetten we de leftmark,
\fancyhead[HR]{\thepage}                % Rechts in de header het paginanummer.
% Activeer de volgende lijn en desactiveer de vorige om paginanummers onderaan gecentreerd te krijgen.
%\fancyhf[FC]{\thepage}                  % Paginanummers onderaan gecentreerd.

% PDF specifieke opties, niet strict noodzakelijk voor een thesis.
% Is hetgeen verschijnt wanneer je in acroread de documentproperties bekijkt.
\hypersetup{
    pdfauthor = {Douwe De Bock},
    pdftitle = {Opgeloste vragen besturingssystemen 3},
    pdfsubject = {Opgeloste examen vragen voor mondeling en schriftelijk examen
    van Windows},
    pdfkeywords = {windows, bs3, tiwi, ldap}
}


% Het volgende commando zou ervoor moeten zorgen dat er een witte ruimte wordt gelaten tussen
% elke paragraaf. Het zorgt ervoor dat er echter teveel witte ruimte komt boven en onder de
% verschillende titels, gemaakt met \section, subsection...
%%\setlength{\parskip}{0ex plus 0.3ex minus 0.3ex}

% Vandaar dat we expliciet aangeven wanneer we wensen dat een nieuwe paragraaf begint:
% \par zorgt ervoor dat er een nieuwe paragraaf begint en
% \vspace zorgt voor vertikale ruimte.
\newcommand{\npar}{\par \vspace{2.3ex plus 0.3ex minus 0.3ex}}

%%%%%%%%%%%%%%%%%%%%%%%%%%%%%%
% Nieuwe commando's
%%%%%%%%%%%%%%%%%%%%%%%%%%%%%%

% De differentiaal operator
\newcommand{\diff}{\ensuremath{\mathrm{d}}} 

% Super en subscript
\newcommand{\supsc}[1]{\ensuremath{^{\text{#1}}}}   % Superscript in tekst
\newcommand{\subsc}[1]{\ensuremath{_{\text{#1}}}}   % Subscript in tekst

% Chemische formule font:
\newcommand{\ch}[1]{\ensuremath{\mathrm{#1}}\xspace}	 
% Chemische pijl naar rechts:
\newcommand{\chpijlr}{\ensuremath{\hspace{1em}\longrightarrow\hspace{1em}}}
% Chemische pijl naar links:
\newcommand{\chpijll}{\ensuremath{\hspace{1em}\longleftarrow\hspace{1em}}}
% Chemische pijl naar links en rechts:
\newcommand{\chpijllr}{\ensuremath{\hspace{1em}\longleftrightarrow\hspace{1em}}}

\newcommand{\vt}[1]{\ensuremath{\boldsymbol{#1}}} % vector in juiste lettertype
\newcommand{\mx}[1]{\ensuremath{\mathsf{#1}}}	  % matrix in juiste lettertype

% Het latex logo in een eenvoudiger commando steken
\newcommand{\latex}{\LaTeX\xspace}

% Het BibTeX logo
\newcommand{\bibtex}{\textsc{Bib}\TeX\xspace}

% Niew commando om bestandnamen anders weer te geven
\newcommand{\bestand}[1]{\lstinline[basicstyle=\sl]{#1}\xspace}

% Niew commando om commando tekst weer te geven
\newcommand{\command}[1]{\lstinline[basicstyle=\tt]{#1}\xspace}
\newcommand{\commandx}[1]{\index{#1}\lstinline[basicstyle=\tt]{#1}\xspace}

%\lstset{morecomment={\%}}
% Commando om latex commando`s weer te geven (x: voor indexing)
%\newcommand{\lcommand}[1]{\lstinline[basicstyle={\tt},{language=[LaTeX]TeX}]{#1}\xspace}
\newcommand{\lcommand}[1]{\lstinline[basicstyle={\tt}]{#1}\xspace}
\newcommand{\lcommandx}[1]{\index{#1}\lstinline[basicstyle=\tt]{#1}\xspace}


% Niew commando om vreemde taal weer te geven (hint: dit commando kan gebruikt
%   worden om latijnse namen, die ook cursief moeten staan, weer te geven.
\newcommand{\engels}[1]{\textit{#1}\xspace}
\newcommand{\engelsx}[1]{\index{#1}\textit{#1}\xspace}

% Niew commando om iets te benadrukken en tegelijkertijd in de index te steken.
\newcommand{\begrip}[1]{\index{#1}\textbf{#1}\xspace}

% Nieuw commando om figuren in te voegen. Gebruik:
% \mijnfiguur[H]{width=5cm}{bestandsnaam}{Het bijschrift bij deze figuur}
\newcommand{\mijnfiguur}[4][ht]{            % Het eerste argument is standaar `ht'.
    \begin{figure}[#1]                      % Beginnen van de figure omgeving
        \begin{center}                      % Beginnen van de center omgeving
            \includegraphics[#2]{#3}        % Het eigenlijk invoegen van de figuur (2: opties, 3: bestandsnaam)
            \caption{#4\label{#3}}          % Het bijschrift (argument 4) en het label (argument 3)
        \end{center}
    \end{figure}
    }

% Nieuw commando om figuren in te voegen. Gebruik:
% \mijnfiguur[H]{bestand-tabular}{Het bijschrift bij deze tabel}    
\newcommand{\mijntabel}[3][ht]{             % Het eerste argument is standaar `ht'.
    \begin{table}[#1]                       % Beginnen van de table omgeving
        \begin{center}                      % Beginnen van de center omgeving
            \caption{#3\label{#2}}          % Het bijschrift (argument 3) en het label (argument 2)
            \input{#2}                      % Invoer van de tabel
        \end{center}
    \end{table}
    }

%%%%%%%%%%%%%%%%%%%%%%%%%%%%%%
% Nieuwe wiskunde operatoren
%%%%%%%%%%%%%%%%%%%%%%%%%%%%%%

\DeclareMathOperator{\integ}{Integraal}

%%%%%%%%%%%%%%%%%%%%%%%%%%%%%%
% Nieuwe omgevingen
%%%%%%%%%%%%%%%%%%%%%%%%%%%%%%

% Een soort theorem omgeving
\newtheorem{levensles}{Levensles}[chapter]

% Om minder belangrijke delen iets kleiner te zetten.
\newenvironment{MinderBelangrijk}{\small}{}

% Een nieuwe omgeving om letterlijke latex tekst weer te geven.
\lstnewenvironment{llt} 
    {
    \vspace{1.2ex plus 0.5ex minus 0.5ex}   % Beetje ruimte voor de letterlijke tekst
    \lstset{                                % Enkele opties:
        basicstyle={\small\tt},             % Iets kleiner
        %language=[LaTeX]{TeX},              % Syntax highlighting
        stepnumber=0,                       % De lijnen worden niet genummerd
        breaklines=true,                    % Als een lijn te lang is, wordt hij afgebroken
        basewidth={0.5em},                  % Breedte van een letter
        xleftmargin=1em}                    % Inspringing van de linker marge
    }
    {\vspace{0.9ex plus 0.5ex minus 0.5ex}  % Beetje ruimte na de letterlijke tekst
    }

% Een nieuwe omgeving om algemene letterlijke tekst weer te geven.
\lstnewenvironment{lt} 
    {
    \vspace{1.2ex plus 0.5ex minus 0.5ex}   % Beetje ruimte voor de letterlijke tekst
    \lstset{                                % Enkele opties:
        basicstyle={\small\tt},             % Iets kleiner en typmachine lettertype
        stepnumber=0,                       % De lijnen worden niet genummerd
        breaklines=true,                    % Als een lijn te lang is, wordt hij afgebroken
        basewidth={0.5em},                  % Breedte van een letter
        xleftmargin=1em}                    % Inspringing van de linker marge
    }
    {\vspace{0.9ex plus 0.5ex minus 0.5ex}  % Beetje ruimte na de letterlijke tekst
    }



%%%%%%%%%%%%%%%%%%%%%%%%%%%%%%
% Einde van de preamble.
% Begin van de body:
%%%%%%%%%%%%%%%%%%%%%%%%%%%%%%

\begin{document}

\frontmatter

\part*{EXAMENVRAGEN Besturingssystemen 3}

\tableofcontents

\mainmatter

% De verschillende hoofdstukken:
\part{Mondeling}
\chapter{Structuur van Active Directory gegevens}

\section{Bespreek de diverse namen die alle Active Directory objecten
identificeren.}

\begin{description}
	\item[Relative Distinguished Name (RDN)] zorgt voor de unieke
		identificatie binnen de container waar het object zich in
		bevindt. bv \begrip{cn=beelzebub} is RDN van computer beelzebub.
		De RDN wordt opgeslagen in het kenmerk \begrip{cn} van het
		object.
	\item[Distinguished Name (DN)] is opgebouwd uit de RDN van het object
		zelf en de RDNs van alle containerobjecten waarvan het object in
		de hiërarchische indeling deel van uitmaakt. Dit zorgt voor een
		unieke naamgeving voor elk object. Elk opeenvolgend deel van de
		DN heeft de vorm \begrip{attribuut=waarde}, staat bekend als
		\engels{attributed naming}. De delen worden gescheiden door een
		komma. De DN naamgeving werkt van beneden (objectnaam) naar
		boven (root van AD). bv
		\begrip{cn=beelzebub,ou=iii,dc=hogent,dc=be}
		De DN is belangrijk voor de werking van het LDAP protocol, bv
		belangrijk bij het programmeren van scriptcode die AD
		rechtstreeks aanspreekt. Een LDAP-cliënt kan query's op
		willekeurige objecten uitvoeren door een LDAP URL te gebruiken
		met de vorm \begrip{LDAP://server DNS naam/object DN}. De DN
		wordt opgeslagen in het kenmerk \begrip{distinguishedName} van
		het object.
	\item[Canonical name] wordt op dezelfde manier samengesteld als de DN,
		maar wordt op een eenvoudigere manier weergegeven:
		\begrip{hogent.be/iii/beelsebub}. De meeste hulpmiddelen in
		Active Directory tonen de canonieke naam, in plaats van de DN.
		Het wordt opgeslagen in het kenmerk \begrip{canonicalName} van
		het object.
	\item[Globaal Uniek ID (GUID)] is een 128-bit getal dat niet kan
		gewijzigd worden. Het wordt bepaald bij de creatie van het
		object. Ook als het object verplaatst of gewijzigd wordt zal de
		GUID niet veranderen. De GUID is beschikbaar voor verwijzing van
		externe processen en programmeerfuncties. Het wordt opgeslagen
		in het kenmerk \begrip{objectGUID} van het object.
\end{description}

\section{Wat zijn SPN objecten? Bespreek de aanvullende naamgeving voor deze
objecten.}

Security Principal Objects (SPN) zijn AD-objecten waaraan Security ID's (SID)
zijn goegewezen. De SID wordt opgeslagen in het \begrip{objectSid} kenmerk.
SPN objecten worden, na aanmelding op het netwerk, gebruikt voor het verlenen
van toegang tot domeinbronnen. Ze zijn van toepassing op gebruikeraccounts,
computeraccounts, groepen en domeinen. SIDs zijn uniek, ook in de tijd: nieuw
aangemaakte accounts kunnen zo nooit de rechten krijgen van een oud account.

SIDs worden meestal voorgesteld dooe een hiërarchische string getallen
gescheiden door koppeltekens. bv \begrip{S-1-5-x-y-z-500} S-1-5 is de standaard
prefix, bestaande uit een \engels{Revision Level} en een \engels{Authority
Identifier}. X, y en z zijn 32-bit getallen specifiek voor het domein
(\engels{Domain Subauthority Identifier}). 500 is een \engels{relative ID}
(RID), dat nar het feitelijk object verwijst.

Bij verplaatsing binnen hetzelfde domein blijft de SID ongewijzigd. Bij
verplaatsing naar een ander domein zal de SID we veranderen. De GUID blijft
altijd ongewijzigd. Om te vermijden dat een gebruiker hierna zijn toegang tot
domeinbronnen zou verliezen worden alle vorige SIDs van het object bijgehouden
in het \begrip{sIDHistory} kenmerk.

\subsection{Aanvullende naamgeving voor gebruikersaccounts}

De RDN, DN en canonieke naam zijn ongeschikt als aanmeldingsnaam omdat ze bij
verplaatsing veranderen. SID en GUID zijn mits hun numerieke karakter ook niet
geschikt hiervoor. Elke gebruikersaccount heeft ook een \engels{User Principal
Name} (UPN), aka aanmeldingsnaam. Wordt bij het aanmaken van het account
ingevoerd door de beheerder en opgeslagen in het \engels{userPrincipalName}
kenmer. Dit moet uniek zijn binnen het forest. UPN bestaat standaard uit de RDN
van de gebruiker en de UPN suffix, gekoppeld met het @-teken.
Voor de UPN suffix zijn er een aantal alternatieven:
\begin{itemize}
	\item Standaard wordt de DNS naam van het domein genomen
	\item Dikwijlst kiest men voor de DNS naar van het root domein van het
		forest
	\item Kan ook een willekeurige alternatieve naam zijn, als deze op
		voorhand geregistreerd is door een beheerder.
\end{itemize}

UPN is de meest geschikte aanmeldingsnaam maar wordt maar zelden gebruikt door
compatibiliteit met vroegere NT versies. Meestal wordt als aanmeldingsnaam de
NetBIOS naam van het domein en de SAM accountnaam van de gebruiker, gekoppeld
met \, genomen. De NetBIOS naam bestaat uit max 15 karakters, standaard de meest
linkse component van de DNS naam. De SAM accountnaam bestaat uit max 20
karakters, standaard de eerste 20 bytes van de RDN. NetBIOS naam moet uniek zijn
in het forest, SAM accountnaam uniek in het domein en wordt opgeslagen in het
\begrip{sAMAccountName} kenmerk

\subsection{Aanvullende naamgeving voor computeraccounts}

Elke computeraccount in AD heeft behalve zijn RDN, DN, canonieke naam, GUID en
SID ook een SAM accountnaam (ook NetBIOS computernaam genoemd), een DNS hostnaam
en een \engels{Service Principal Name} (SPN).
SAM naam bestaat uit eerste 15 bytes van RDN, gevolgd door een \$, en wordt
opgeslagen in het sAMAccountName kenmerk.
De DNS hostname wordt opgeslagen in het dnsHostName kenmerk en bestaat
standaard uit de eerste 15 tekens van de RDN en de suffix van de primaire DNS.
De SPN is essentieel tijdens de wederzijdse verificatie van client software en
de server die een bepaalde service biedt: de client zoekt een computeraccount op
aan de hand van de SPN naam van de service waarmee hij een verbinding tot stand
wil brengen. De SPN wordt bepaald door het multi-valued
\engels{servicePrincipalName} kenmerk, oa samengesteld op basis van DNS naame en
eventueel SRV records die naar het toestel verwijzen.

\section{Enkele veel gebruikte klassen vertonen nog meer identificerende
attributen voor hun instanties. Bespreek deze klassen en attributen}

\begin{itemize}
	\item Common name (cn)
	\item IDAPDisplayName
	\item Object identifier (OID)
\end{itemize}

\section{In welke partities is de Active Directory informatie verdeeld? Geef de
betekenis van elke partitie, hun onderlinge relatie, en de
replicatiekarakteristieken erven.}

De directory wordt enkel opgeslagen op domeincontrollers. Op elke
domeincontroller wordt een kopie opgeslagen van de directory voor het domein
waarin de controller zich bevindt. De informatie die in de directory is
opgeslagen, is fysiek verdeeld in minimaal drie categorieën: \begrip{directory
partities}
\begin{description}
	\item[Domaingegevens] bevatten de eigenlijke informatie over objecten in
		het domein (zoals gedeelde bronnen, gebruiker- en
		computeraccounts). Bij de installatie worden een aantal
		standaard objecten geinstalleerd door het systeem, bv de
		beheerderaccount. Wordt gedeeld per domein, dus er zijn evenveel
		partities met domeingegevens als er domeinen zijn in het forest.
		De domeingegevens worden niet gedistribueerd naar andere
		domeinen. Er wordt we een subset van alle kenmerken van alle
		domeinen bijgehouden in de globale catalogus.
	\item[Configuratiegegevens] beschrijven de fysieke topologie van de
		directory. Dit bevat onder andere een lisjt van alle
		domeinstructuren, de locaties van de domeincontrollers en de
		global catalog controllers, de sites en de replicatietopologie.
		De meeste gemeenschappelijke instellingen van het ganse forest
		worden hierin opgeslagen, als kenmerken van objecten. De
		configuratiegegevens gelden voor alle domeinen in het forest.
	\item[Het schema] is de formele definitie van alle objecten en
		kenmerkgegevens die kunnen opgeslagen worden in de directory.
		Dit schema is uniek voor alle domeinen in het forest.
	\item[Applicatiepartities] dit is een 4e vorm van partities die vanaf
		Windows server 2003 kan aangemaakt worden. Een deel van de AD
		gegevens kan in één of meerdere gescheiden applicatiepartities
		worden ondergebracht, dit is interessant voor dynamische
		objecten. Ze kunnen geen SPN objecten bevatten.  Objecten binnen
		de applicatie partitie kunnen niet verplaatst worden buiten de
		partitie. bv voor DNS is er een aparte applicatiepartitie.
\end{description}

Het schema en de configuratiegegevens zijn containerobjecten in het root domein
van het forest. Ook de eventuele applicatiepartities vormen containerobjecten
ten opzichte van de overeenkomstige domeingegevens. Dit is een logische
structurering, fysische bestaat het wel uit meerdere bestanden.

AD servers wisselen continu gegevens uit. Elke partitie in de directory vormt
een aparte eenheid voor replicatie, waarbij telkens een specifieke groep
controllers hoort. Het schema en configuratiegegevens worden gerepliceerd naar
alle domeincontrollers in het forest. De domeingegevens worden logischerwijs
binnen het domein gerepliceerd. Applicatiepartities worden gerepliceerd tussen
een eigen deelverzameling specifiek hiervoor geconfigureerde domeincontrollers
van het forest. DNS partities kunnen zo bv enkel naar DNS nameservers
gerepliceerd worden.

De koppeling tussen een partitie en zijn replicerende domeincontrollers wordt
bijgehouden in het \begrip{msDS-NC-Replica-Locations} kenmerk van het
overeenkomstige crossref object in de configuratiegegevens.

Een subset van de kenmerken van alle objecten in de domeingegevens van elk
domein in het forest worden gerepliceerd naar de globale catalogus.

\chapter{attributeSchema objecten}

\section{Bespreek het doel en de werking van attributeSchema objecten. Hoe
kunnen deze objecten het best geraadpleegd en gewijzigd worden?}

Het AD schema is de formele definitie van alle objecten en kenmerkgegevens die
kunnen opgeslagen worden in de directory. Het is een set regels waarmee de
klassen van objecten en kenmerken in de directory, de beperkingen en limieten op
exemplaren van deze objecten en de notatie van de namen van de objecten
gedefinieerd worden. Deze definities zelf worden als objecten opgeslagen in de
schema container, ze kunnen zo op eenzelfde manier beheerd worden als alle AD
objecten. Er zijn twee types definities:
\begin{description}
	\item[Kenmerken] worden apart van klassen gedefinieerd.
	\item[Klassen] beschrijven de directory objecten die gemaakt kunnen
		worden. Elke klasse heeft een verzameling kenmerken.
\end{description}

\subsection{Doel en werking}

Een attributeSchema object is een object waarmee een kenmerk wordt ingesteld en
waarmee beperkingen opgelegd worden aan objecten die een exemplaar zijn van de
klasse met dit kenmerk. Elk kenmerk moet zo maar eenmaal gedefinieerd worden,
maar kan toch in meerdere klassen gebruikt worden.

\subsection{Raadplegen en wijzigen}

Voor ontwikkelings- en testdoeleinden kun je het AD schema bekijken en wijzigen
met adsiedit.msc, of met een specifiek hiervoor ontwikkelde snap-in, Active
Directory Schema. Er is geen standaard MMC console die deze snap-in bevat. Het
wordt geimplementeerd door schmmgmt.dll, welke na installatie nog moet
geregistreerd worden met regsvr32. De Schema snap-in laat toe, op een meer
eenvoudige wijze dan adsiedit.msc, om zowel kenmerken als klassen te bekijken,
te wijzigen en te creëeren. AD ondersteunt geen verwijdering van schemaobjecten.
Deactiveren is wel mogelijk, toch voor eigen ontwikkelde schemaobjecten, niet
voor de standaard meegeleverde objecten.

\section{Bespreek de diverse naamgevingen van attributeSchema objecten}

Alle attributeSchema objecten in het schema hebben een viervoudige naamgeving,
die alle vier uniek en gestandaardiseerd zijn:
\begin{description}
	\item[common name (cn)] is de RDN van het attributeSchema object in de
		Schema container. Opgeslagen in een kenmerk van het
		attributeSchema object met als LDAP display naam \begrip{cn}.
	\item[GUID] van het kenmerk is onafhankelijk van de GUID van het
		attributeSchema object en kan automatisch gegenereerd worden bij
		creatie van een nieuw kenmerk. Hetzelfde kenmerk zal dan wel een
		ander GUID hebben in verschillende forests. Om dit te vermijden
		kan best op voorhand een GUID gegenereerd worden. Opgeslagen in
		een kenmerk van het attributeSchema object met als LDAP display
		naam \begrip{schemaIDGUID}.
	\item[LDAP display name] is belangrijk voor programmatische toegang.
		Deze kan dikwijls, maar niet altijd, uit de common name afgeleid
		worden door alle streepjes te verwijderen, en de eerste letter
		niet in hoofdletters te vermelden. Opgeslagen in een kenmerk van
		het attributeSchema object met als LDAP display naam
		\begrip{IDAPDisplayName}.
	\item[Object identifier (OID)] die geldt als interne representatie.
		X.500 IDs worden verleend door speciale autoriteiten, en zijn
		gegarandeerd uniek in alle netwerken wereldwijd. Het is een
		decimale reeks met punten, en worden hiërarchisch toegekend. Je
		kan een OID tak aanvragen bij de regionale ISO vertegenwoordiger
		of je kan een uniek OID laten genereren in een microsoft subtak
		met oidgen. Opgeslagen in een kenmerk van het attributeSchema
		object met als LDAP display naam \begrip{attributeID}.
\end{description}

\section{Bespreek de belangrijste kenmerken van attributeSchema objecten, en hoe
die ingesteld kunnen worden.}

\begin{description}
	\item[attributeSyntax en oMSyntax] De syntax van het kenmerk bepaalt het
		data type, en zo het soort gegevens dat het kenmerk kan
		bevatten. Er zijn 26 mogelijkheden waarvan er maar 18 momenteel
		gebruikt worden in AD. Het is niet mogelijk om een nieuwe syntax
		te definiëren. Omdat men in AD bepaalde noodzakelijke data types
		niet van elkaar kan onderscheiden op basis van louter de X.500
		syntax is er een aanvullende integer waarde voorzien: OMSyntax.
	\item[rangeLower en rangeUpper] Bepalen lengte- of bereikbeperkingen van
		kenmerken.
	\item[isSingleValued] Kenmerken kunnen, afhankelijk van dit kenmerk, één
		waarde of meerdere niet-geordende waarden hebben. bv objectClass
		bevat de specifieke klasse van het object en de opeenvolgende
		klassen waarvan de klasse is afgeleid.
	\item[searchFlags] Dit bevat binaire informatie, waarbij de meeste bits
		bepalen of het kenmerk op een of andere manier geïndexeerd
		wordt. Als je een kenmerk indexeert, kun je in Active Directory
		sneller objecten zoeken die dat kenmerk hebben. Alle exemplaren
		van het kenmerk worden dan toegevoegd aan de index, niet alleen
		de exemplaren die lid zijn van een bepaalde klasse.
		\begin{itemize}
			\item De laagste bit wordt meestal gezet en bepaalt
				eenvoudige indexering van de waarde van het
				kenmerk.
			\item De tweede laagste bit zorgt voor een containerized
				index en zorgt dat objecten snel kunnen gevonden
				worden binnen een specifieke container. De
				waarde van het kenmerk wordt hiervoor
				gecombineerd met de identificatie van de
				container.
			\item Het zetten van de derde laagste bit laat ambiguous
				name resolution (ANR) toe. Bij opzoekingen waar
				men zoek gaat naar objecten waarbij minstens één
				kenmerk uit een verzameling kenmerken een
				specifieke waarde aanneemt. Bv voor displayName,
				givenName \& name staat de ANR bit op 1.
			\item Instellen van de zesde laagste bit versnelt
				opzoekingen waarin kenmerken met jokertekns
				vermeld worden.
		\end{itemize}
		De vijfde laagste bit heeft niks met indexering te maken en
		bepaalt of de waarde van attribuut behouden blijft bij het
		kopieren van het object.
	\item[systemFlags] is een binair informatieveld. De laagste bit bepaalt
		of het kenmerk gerepliceerd wordt naar andere domeincontrollers.
		Niet-gerepliceerde kenmerken worden dikwijls gebruikt om lokale
		caches te implementeren. Wordt ook gebruikt voor relatief
		dynamische kenmerken waarvan de waarde frequent wijzigt, zoals
		lastLogon en LastLogoff. LastLogonTimestamp wordt wel
		gerepliceerd.
		De derde laagste bit wijst op een geconstrueerd attribuut. Een
		geconstrueerd attribuut wordt niet opgeslagen in AD, maar wordt
		telkens opnieuw berekend. Bv canonicalName en parentGUID
	\item[isMemberOfPartialAttributeSet] bepaalt of het kenmerk in de global
		catalog opgenomen wordt of niet.
	\item[linkID] Sommige kenmerken vormen koppels bestaande uit
		forward-link en back-link kenmerken. Indien de waarde van het
		forward-link kenmerk van een object verwijst naar de DN van een
		ander object, dan wordt het back-link kenmerk van dat object
		automatisch in- of aangevuld met de DN van het eerste object, en
		vice versa. Gebruikers met voldoende machtigingen kunnen enkel
		de waarde van forwart-link kenmerken rechtstreeks wijzigen.
		Back-link kenmerken vallen volledig onder het beheer van de
		security accounts manager component van windows server.
\end{description}

\section{Welke andere types objecten bevat het Active Directory schema, en wat
is hun bedoeling?}

\begin{description}
	\item[classSchema objecten] Net zoals voor kenmerken bevindt zich voor
		alle klassen een classSchema object in het schema. De kenmerken
		van classSchema objecten definieren de klasse, en bevatten twee
		soorten regels: structuurregels definieren de hiërarchische
		relaties tussen hetzij klassen, hetzij tussen objecten,
		inhoudsregels kenmerken definiëren die beschikbaar zijn voro een
		exemplaar van die klasse.
	\item[het abstracte schema] Naast classSchema en attributeSchema
		objecten bevat het schema nog één ander object: een subSchema
		object, het abstracte schema genoemd. Het heeft als RDN
		Aggregate en bevat een alternatieve, compacte voorstelling van
		het gehele schema. De bedoeling is om vereenvoudigde schema
		gegevens ter beschikking te stellen aan LDAP cliënts, zonder
		zich over veel implementatie details hoeven te bekommeren.
		Gecombineerd met de Active Directory Service Interfaces (ADSI)
		biedt het abstracte schema een toegang naar het schema, die veel
		meer high-level is dan via het reële schema.
\end{description}

\section{Via welke attributen kun je de klasse van een willekeurig Active
Directory object achterhalen? Hoe moet je op zoek gaan naar alle objecten van
een bepaalde klasse? Illustreer aan de hand van relevante voorbeelden.}

\begin{description}
	\item[ObjectClass] is multi-valued en niet geïndexeerd. het bevat niet
		alleen de klasse van het object zelf, maar ook alle
		hiërarchische superklassen (uitgezonderd de statische
		hulpklassen).
	\item[objectCategory] is single-valued en is geïndexeerd. Het wordt
		echter niet noodzakelijk ingevuld met de klasse van het object,
		het bevat de meest typische vertegenwoordiger uit de verzameling
		bestaande uit de klasse zelf en alle hiërarchische superklassen.
\end{description}

\subsection{voorbeelden}

Voor het opzoeken van printers is de selectie van objecten, waarvoor de
objectCategory ingesteld is op printQueue, duidelijk de beste keuze, aangezien
dit de opzoeking toelaat om op indexering een beroep te doen.

Indien men analoog gebruikers zou willen opzoeken via objecten, waarvoor de
objectCategory ingesteld is op person, dan is dit weliswaar een performante
oplossing, maar zal dit niet alleen objecten van de klasse user opleveren, maar
ook objecten van de klasse contact. Om deze uit te sluiten, zou men kunnen
overwegen om objecten te selecteren, waarvoor user tot de objectClass behoort.
Dit heeft echter het nadeel dat de zoekopdracht nu niet alleen objecten van de
klasse user zal opleveren, maar ook objecten van de ervan afgeleide klasse
computer. De enige juiste mogelijkheid in dit geval is om op zoek te gaan naar
objecten waarvoor zowel de objectCatergory ingesteld is op person, als user tot
de objectClass behoort.

\chapter{classSchema objecten}

\section{Bespreek het doel en de werking van classSchema objecten}

Klassen zijn zelf ook objecten in het schema: voor elke klasse in het schema is
er een classSchema object waarmee de klasse ingesteld wordt. Ze beschrijven de
directory objecten die gemaakt kunnen worden.
De kenmerken van classSchema objecten definiëren de klasse, en bevatten twee
soorten regels: met structuurregels definieer je de mogelijke hiërarchische
relaties tussen hetzij klassen, hetzij tussen objecten, terwijl inhoudsregels
kenmerken definiëren die beschikbaar zijn voor een exemplaar van die klasse.
Door middel van overname kunnen van bestaande klassen nieuwe klassen aangemaakt
worden.

\section{Hoe benadert Active Directory het mechanisme van overerving?}

Door overerving kan je van bestaande klassen nieuwe klassen maken. De
onmiddelijke superklasse van een klasse wordt bepaald door het kenmerk
subClassOf van het classSchema object. Speciale superklasse Top waar alle
klassen (onrechtstreeks) van worden afgeleid. Een subklasse neemt de kenmerken
van de superklasse over, inclusief de structuurregels en de inhoudsregels.
Overname werkt recursief.
Een klasse kan enkel kenmerken overnemen van zijn onmiddelijke superklasse en
van speciaal hiervoor bestemde hulpklassen. Hulpklassen zijn klassen die zelf
geen kenmerken kunnen genereren.
De kenmerken auxiliaryClass en systemAuxiliaryClass van elk classSchema object
bevatten alle mogelijke klassen waarvan deze klasse kenmerken kan overnemen.
Kenmerken met een LDAP display naam die beginnen met system zijn in
tegenstelling tot de corresponderende kenmerken niet wijzigbaar door beheerders
van het schema.
Het gebruik van hulpklassen kan zowel statisch (met auxiliaryClass kenmerk) als
dynamisch (hulpklasse opgeven bij creatie) gebeuren.

\section{Bespreek de diverse naamgevingen van classSchema objecten}

Alle classSchema objecten in het schema hebben een viervoudige naamgeving, die
alle vier uniek en gestandaardiseerd zijn:
\begin{description}
	\item[common name] is de RDN van het classSchema object in de schema
		container. Opgeslagen in een kenmerk van het classSchema object
		met als LDAP display naam \begrip{cn}.
	\item[GUID] van de klasse is onafhankelijk van de GUID van het
		classSchema object en kan automatisch gegenereerd worden bij
		creatie van een nieuwe klasse. Opgeslagen in een kenmerk van het
		classSchema object met als LDAP display naam
		\begrip{schemaIDGUID}.
	\item[LDAP display name] is belangrijk voor programmatische toegang.
		Deze kan dikwijls, maar niet altijd, uit de common name afgeleid
		worden door alle streepjes te verwijderen, en de eerste letter
		niet in hoofdletter te vermelden. Opgeslagen in een kenmerd van
		het classSchema object met als LDAP display naam
		\begrip{IDAPDisplayName}.
	\item[Object ID] die geldt als interne representatie.
		X.500 IDs worden verleend door speciale autoriteiten, en zijn
		gegarandeerd uniek in alle netwerken wereldwijd. Het is een
		decimale reeks met punten, en worden hiërarchisch toegekend. Je
		kan een OID tak aanvragen bij de regionale ISO vertegenwoordiger
		of je kan een uniek OID laten genereren in een microsoft subtak
		met oidgen. Opgeslagen in een kenmerk van het classSchema
		object met als LDAP display naam \begrip{governsID}.
\end{description}

\section{Bespreek de belangrijkste kenmerken van classSchema objecten, en hoe
die ingesteld kunnen worden.}

De kenmerken van classSchema objecten definiëren de klasse, en bevatten twee
soorten regels: met structuurregels definieer je de mogelijke hiërarchische
relaties tussen hetzij klassen, hetzij tussen objecten, terwijl inhoudsregels
kenmerken definiëren die beschikbaar zijn voor een exemplaar van die klasse.

\subsection{Inhoudsregels}

\begin{description}
	\item[mustContain en systemMustContain] kenmerken bevatten de lijst
		met kenmerken die verplicht zijn in elk exemplaar van de klasse.
		Een kenmerk is verplicht van zodra het verplicht is in één van
		de hiërarchische superklassen van de klasse, ook al is het voor
		de klasse zelf als optioneel gemarkeerd.
	\item[mayContain en systemMayContain] kenmerken bevatten de lijst met
		kenmerken die optioneel zijn.
	\item[rDNAttID] kenmerk bepaalt welk kenmerk van een klasse gebruikt
		wordt om de RDN van objecten te bepalen. Meestal staat dit
		kenmerk ingesteld op de waarde cn (common name).
	\item[defaultSecurityDescriptor] kenmerk bepaalt de expliciete
		machtigingen die gelden voor objecten van deze klasse. Dit kan
		een eenvoudige oplossing bieden om het beheer van specifieke
		objecten te delegeren.
	\item[SystemOnly] als dit de waarde True heeft, kunnen de
		structuurregels en de inhoudsregels van de klasse niet gewijzigd
		worden.
	\item[isDefunct] hiermee kunnen schema objecten gedeactiveerd
		worden. Zo kunnen er geen nieuwe exemplaren van de klasse
		aangemaakt worden. AD ondersteunt geen verwijdering van
		schemaobjecten.
\end{description}

\subsection{Structuurregels}
\begin{description}
	\item[objectClassCategory] kenmerk heeft een integer waardie die de
		categorie van de klasse bepaalt: structurele klasse (1),
		abstracte klasse (0 of 2) of hulpklasse (3). Abstracte klassen
		zijn gelijkaardig aan structurele klassen, maar kunnen zelf geen
		objecten genereren.
	\item[defaultObjectCategory]
	\item[subClassOf] kenmerk bepaald de onmiddelijke superklasse van een
		klasse.
	\item[auciliaryClass en systemAuxiliaryClass] kenmerken bevatten alle
		mogelijke hulpklassen waarvan deze klasse kenmerken kunnen
		overnemen.
	\item[possSuperiors en systemPossSuperiors] kenmerken van elk
		classSchema object definieren de mogelijke hiërarchische
		relaties tussen objecten van een klasse. Het feit of een klasse
		containerobjecten representeert, wordt niet bepaald door een
		kenmerk van de klasse zelf: een structurele klasse kan
		containerobjecten genereren van zodra een andere structurele
		klasse ernaar verwijst in zijn systemPossSuperiors of
		PossSuperiors kenmerken. De definitie van een klasse bepaalt
		voor zijn objecten bijgevolg niet van welke klasse het objecten
		kan bevatten als containerobject, maar wel van welke klasse de
		objecten als container kunnen optreden.
\end{description}

\section{Welke andere types objecten bevat het Active Directory schema, en wat
is hun bedoeling?}

\begin{description}
	\item[attributeSchema objecten] zijn objecten waarmee een kenmerk wordt
		ingesteld en waarmee beperkingen opgelegd worden aan objecten
		die een exemplaar zijn van de klasse met dit kenmerk. Elk
		kenmerk moet zo maar eenmaal gedefinieerd worden, maar kan toch
		in meerdere klassen gebruikt worden.
	\item[Abstracte schema] Naast classSchema en attributeSchema
		objecten bevat het schema nog één ander object: een subSchema
		object, het abstracte schema genoemd. Het heeft als RDN
		Aggregate en bevat een alternatieve, compacte voorstelling van
		het gehele schema. De bedoeling is om vereenvoudigde schema
		gegevens ter beschikking te stellen aan LDAP cliënts, zonder
		zich over veel implementatie details hoeven te bekommeren.
		Gecombineerd met de Active Directory Service Interfaces (ADSI)
		biedt het abstracte schema een toegang naar het schema, die veel
		meer high-level is dan via het reële schema.
\end{description}

\section{Hoe en met welke middelen kan het Active Directory schema uitgebreid
worden?}

Uitbereidingen en wijzigingen van het schema zijn risicovol, geldne voor het
hele forest en kunnen potentieel de hele infrastructuur onbruikbaar maken.
Schema objecten worden daarom ook beveiligd met ACL's. Aanmaken van geheel
nieuwe structurele klassen en wijzigen van attributen van bestaande klassen
moeten zoveel mogelijk vermeden worden.

Kleinschalige wijzigingen kunnen gebeuren met de Active Directory Schema
snap-in. Een veilge manier om attributen aan een klasse toe te voegen: maak
eerst de nodige attributeSchame objecten aan. Vervolgens wordt een nieuwe
hulpklasse aangemaakt, waarin de lijst van optionele attributen wordt aangevuld
met de nieuw aangemaakte attributen. Tenslotte wordt de nieuw aangemaakte
hulpklasse geassocieerd met de klasse waaraan we de attributen wouden toevoegen.

Grootschalige uitbereidingen gebeuren best met ldifde of programmatisch met ADSI
interfaces. Met ldifde kunnen bestanden die geformatteerd zijn in LDAP Data
Interchange Format worden toegepast op de directory. Om het schema aan te
passen, stellen we alle nodige acties voor in het LDIF formaat. Het bestand
wordt dan meegegeven aan ldifde -i -f.

\chapter{Active DIrectory functionele niveaus}

\section{Geef de diverse functionele niveaus waarop Active Directory kan
ingesteld worden, en welke beperkingen er het gevolg van zijn.}

Domein functioneel niveau
\begin{itemize}
	\item Niveau 0: Windows 2000 mixed
	\item Niveau 1: Windows 2000 native
	\item Niveau 2: Windows Server 2003
	\item Niveau 3: Windows Server 2008
	\item Niveau 4: Windows Server 2008 R2
	\item Niveau 5: Windows Server 2012
\end{itemize}

Forest functioneel niveau
\begin{itemize}
	\item Niveau 0: Windows 2000
	\item Niveau 2: Windows 2003
	\item Niveau 3: Windows 2008
	\item Niveau 4: Windows 2008 R2
	\item Niveau 5: Windows 2012
\end{itemize}

\section{Bespreek van elk niveau alle eraan gekoppelde voordelen. Geef hierbij
telkens een korte bespreking van de ingevoerde begrippen.}

Domein functioneel niveau
\begin{description}
	\item[Niveau 1: Windows 2000 native] 1 global catalog voor heel forest,
		Automatisch wederzijdse vertrouwensrelaties tussen alle domeinen
		van een forest, Alle domeincontrollers kunnen zelfstandig een
		aantal SPN-objecten aanmaken
	\item[Niveau 2: Windows Server 2003] Aanvullende schema klassen en
		attributen
	\item[Niveau 3: Windows Server 2008] Kerberos encryptie met langere
		sleutels, Fine-grained password policies
\end{description}

Forest functioneel niveau
\begin{description}
	\item[Niveau 0: Windows 2000] Stelt geen enkele eis aan het domein
		functioneel niveau van het liddomein
	\item[Niveau 2: Windows 2003] Hergebruik van gedeactiveerde
		schemaobjecten, Dynamisch gebruik van hulpklassen
	\item[Niveau 3: Windows 2008] Geen site coverign probleem bij sites met
		enkel RODC's
	\item[Niveau 4: Windows 2008 R2] Online restore mogelijkheid van
		thombstone objects
\end{description}

\section{Hoe kan men detecteren op welk niveau een Active Directory omgeving
zicht bevindt?}

\begin{itemize}
	\item Domein functioneel niveau
	\item Wordt ingesteld op het domeinobject zelf
	\item Wordt bepaald door twee attributen:
		\begin{itemize}
			\item ntMixedDomein
			\item msDS-Behavior-Version
		\end{itemize}
\end{itemize}

\begin{itemize}
	\item Forest functioneel niveau
	\item Wordt ingesteld op het partitions containerobject van de
		configuratiegegevens
	\item Wordt bepaald door slechts één attribuut: msDS-Behavior-Version
\end{itemize}

\section{Op welk diverse manieren kan men het functionele niveau verhogen of
verlagen?}

De omschakeling naar een bepaald domein of ferst functioneel niveau moet manueel
gebeuren, gebeurt niet automatisch van zodra de controllers of domeinen aan een
minimum voorwaarde voldoen.
\begin{itemize}
	\item Door zelf manueel de attributen te manipuleren
	\item Via de Active Directory Domains and Trust snap-in
\end{itemize}
Men kan enkel het functioneel niveau verhogen, niet verlagen. De wijzigingen
worden pas effectief doorgevoerd na een heropstart van alle domeincontrollers.
Achteraf geinstalleerde domeincontrollers zullen automatisch functioneren op het
huidige niveau.

\chapter{Active Directory domeinstructuren}

\section{Wat is de bedoeling van vertrouwensrelaties?}

Tussen 2 domeinen kan een vertrouwensrelatie tot stand gebracht worden, zodat de
gebruikers in het ene, trusted domein kunnen geverifieerd worden door de
domeincontroller in het andere, trusting domein. Vertrouwensrelaties worden
weergegeven met een pijl in de richting van het trusted domein. Een gebruiker
kan pas toegang krijgen tot bronnen in een ander domein als er een
vertrouwenspad is van het trusting domein naar het trusting domein. Een
vertrouwenspad is een continue rij vertrouwensrelaties tussen domeinen. Dat een
gebruiker door een trusting domeincontroller is geverifieerd, wil nog niet
automatisch zeggen dat de ggebruiker toegang heeft tot de bronnen in dat domein.
Deze toegang wordt geregeld met gebruikersrechten en machtigingen die aan de
gebruiker toegekend zijn door de domeinbeheerder van het trusting domein.

\section{Bespreek de verschillende soorten vertrouwensrelaties.}

\subsection{Automatische vertrouwensrelaties}

Windows Server maakt automatisch vertrouwensrelaties aan tussen domeinen en hun
child domeinen. Deze vertrouwensrelaties kunnen niet verbroken worden en zijn
automatisch bi-directioneel en transitief. Windows Server maakt ook automatisch
vertrouwensrelaties aan tussen de trees van eenzelfde forest: de root domeinen
van alle trees in het forest vormen transitieve vertrouwensrelaties met het
forest root domein van het forest.
Aangezien Windows Server vertrouwensrelaties bi-directioneel en transitief zijn,
heeft een domein dat nieuw aangemaakt wordt in een tree of forest, automatisch
vertrouwensrelaties met alle andere Windows Server domeinen in de tree of het
forest.
Om in NT 4 hetzelfde te bekomen, moet je als systeembeheerder zelf
vertrouwensrelaties construeren, en dan nog in twee richtingen, dit met elk
bestaand domein.

\subsection{Expliciete vertrouwensrelaties}

Transitieve vertrouwensrelaties kunnen alleen bestaan tussen Windows Server
domeinen in hetzelfde forest, tenzij de diverse forests minimaal op Windows
Server 2003 functioneel niveau staan. In dat geval kun je manueel tussen de root
domeinen van de forests bi-directionele en transitieve forest trusts
configureren, waardoor je een gederatie of ream van forests krijgt. Elk koppel
domeinen in een dergelijke realm vertrouwt elkaar wederzijds. Bij meerdere
forests moet men forest trusts configureren tussen elk koppel forests.
Realm trusts, zijn een veralgemening van forest trusts, die vertrouwenspaden
leggen tussen Windows Server 2008 domeinen en willekeurige Kerberos v5 realms.
Een realm trust kan zowel bi-directioneel als enkelvoudig, en zowel transitief
als niet-transitief gedefinieerd worden.
Forest en realm vertrouwensrelaties zijn expliciete vertrouwensrelaties: trusts
die je zelf maakt, in tegenstelling tot de vertrouwensrelaties die automatisch
gemaakt worden tijdens de creatie van het domein.
Een verkorte vertrouwensrelatie is ook een expliciete vertrouwenrelatie en
maakt het mogelijk om een vertrouwenspad tussen 2 domeinen, in grote en complexe
trees, te verkorten. Dit moet het aanmeldingsproces versnellen.
Het laatste type expliciete vertrouwensrelatie is de externe vertrouwensrelatie.
Dit is een enkelvoudige vertrouwingsrelatie waarbij één domein een ander domein
vertrouwt. Deze zijn niet-transitief. Je kan geen externe relatie instellen
tussen AD domeinen in hetzelfde forest (heeft al automatische relaties). Dit kan
wel ingesteld worden tussen:
\begin{itemize}
	\item individuele domeinen in een ander forest.
	\item met NT 4 domeinen
\end{itemize}

\section{Op welke diverse manieren kunnen vertrouwensrelaties gecreëerd en
gecontroleerd worden? Bespreek ook de optionele configuratiemogelijkheden}

Enkel de expliciete vertrouwensrelaties moeten manueel geconfigureerd worden.
Dit kan op twee verschillende manieren geconfigureerd worden. Als je een
expliciete vertrouwenrelatie wil maken, moet je beschikken over de domeinnamen
en een gebruikersaccount met machtiging om vertrouwensrelaties in beide domeinen
te maken. Elke vertrouwensrelatie krijgt een wachtwoord teogewezen, dat bekend
moet zijn bij de beheerders van beide domeinen van de vertrouwensrelatie. Na het
opzetten van de vertrouwensrelatie wordt dit wachtword niet meer gebruikt.

\subsection{Active Directory Domains en Trust snap-in}

Deze snap-in is beschikbaar in domein.msc. Het aanmaken van de
vertrouwensrelatie kan door met de rechmuisknop op een domein te klikken en
achtereenvolgens Properties en de Trusts tabpagina te selecteren, en daarna de
New Trust wizard op te starten.

\subsection{Via de command-line}

In de command prompt kan je een vertrouwensrelatie configureren met de netdom
trust opdracht. Met netdom query trust krijg je een overzicht van alle
vertrouwensrelaties en hun toestand.

\subsection{Optionele configuratiemogelijkheden}
\begin{itemize}
	\item Standaard worden alle gebruikers van het trusted domein opgenomen
		in de Authenticated Users impliciete groep van het trusting
		domein. Men kan echter ook voor selective authentication kiezen,
		waardoor dit per individuele gebruiker of gebruikersgroep
		expliciet moet ingesteld worden.
	\item Indien men gebruik maakt van SID Filtering, dan wordt enkel
		rekening gehouden met de SID opgeslagen in het objectSid
		attribuut van de objecten in het trusted domein. Indien men SID
		Filtering uitschakelt, dan verwerkt het trusting domein ook de
		SIDs opgeslagen in het sIDHistory attribuut. Malfide beheerders
		in het trusted domein kunnen langs deze weg zichzelf meer
		machtigingen en rechten toeeigenen in het trusting domein.
\end{itemize}

\section{Welke verschillen zijn er in praktijk tussen NT 4.0 en Windows Server
domeinstructuren? Bespreek de alternatieve mogelijkheden bij de conversie van
een NT 4.0 domeinstructuur naar een Windows Server omgeving.}

Een eerste verschil tussen NT4 en Windows Server is het conceptueel onderscheid
tussen master domeinen en resource domeinen. Een master domein bevat gebruikers
en groepen, terwijl een resource domein lidservers bevat die diensten aanbieden
aan de gebruikers, en zelf nauwelijks gebruikers bevat. De NT4 domeinstructuren
bestaan meestal uit één (of meerdere) master domeinen, en meerdere resource
domeinen. Er worden bidirectionele vertrouwensrelaties aangemaakt tussen alle
masterdomeinen onderling, en unidirectionele vertrouwensrelaties waarbij elk
resourcedomein elk masterdomein vertrouwt.

De omschakeling van NT4 naar Windows Server moet geleidelijk en gefaseerd
gebeuren. Het aantal domeinen moet hierbij dalen. De organizational units (OUs)
van AD vervangen het conceptuele onderscheid tussen resource- en masterdomeinen.
De upgrade begint steeds bovenaan in de domeinhiërarchie: het masterdomein
krijgt als eerste een upgrade, daarna volgen de resource domeinen.

Bestaande domeinen kunnen worden gesimuleerd door OUs in het Windows Server
Domein. We bekomen zo een getrouwe weerspiegeling van de oude structuur.
Eventueel kunnen aanvullende structuren worden toegevoegd om een meer
gedetaileerde ordening te verkrijgen. Op deze manier wordt het aantal domeinen
en trusts gereduceerd.

Indien er bepaalde bedrijfseenheden als afzonderlijke organisaties moeten worden
behandeld, is een forest met afzonderlijke trees een goede oplossing. De
gebruikersaccounts worden dan verplaatst naar de domeinen met de bronnen die ze
gebruiken, in plaats van het centrale root domein.

Indien de oude structuur meerdere NT4 master domeinen bevat, zijn er hiervoor
een aantal mogelijke oorzaken. In eerste instantie kan het zijn dat een enkel
master domein te veel gebruikers en groepen zou bevatten. Deze veroorzaken
immers instabiliteit van de SAM databank. Dit probleem is meteen opgelost door
AD, omdat het veel schaalbaarder is. Een andere mogelijk oorzaak is de
geografische situering. Het is mogelijk dat de verschillende geografische
locaties verbonden zijn door links met kleine bandbreedte. Dit probleem kan
worden opgevangen door AD sites te configureren. Het replicatie verkeer kan nog
verder worden beperkt door gebruik te maken van RODC's. Een derede mogelijkheid
is de noodzaak aan een versschillende wachwoordbeleid voor bepaalde
gebruikersgroepen. Dit kan worden opgevangen door het domein functioneel niveau
te verheffen naar Windows Server 2008, en een fine grained password policy in te
stellen. Als laatste oorzaak voor verschillende NT4 master domeinen kan het zijn
dat bepaalde onderdelen van de organizatie controle moeten kunnen uitoefenen
over eigen bronnen en gebruikers. Dit is bij de invoering van AD de enige reden
om aparte domeinen te behouden in de verschillende sites. De migratie gebeurt op
een vand de volgende manieren:
\begin{enumerate}
	\item Elk NT4 domein wordt geupgraded naar de root van een Windows
		Server tree in hetzelfde forest. Alle gemachtigde gebruikers
		krijgen hierdoor potentiële toegang tot alle bronnen in alle
		domeinen van het forest. Het forest kan een gemeenschappelijk
		schema, gemeeschppelijke configuratiegegevens en een
		gemeenschappelijke global catalog delen.
	\item Als alternatief kan elk NT4 domein woren geupgraded naar een
		subdomein van een artificieel root domein van dezelfde tree. Dit
		root domein heet een structural of placeholder domein omdat het
		resources noch accounts bevat. Het is echter wel een uistekende
		plaats om de global catalog onder te brengen.
\end{enumerate}

Als laatste mogelijkheid voor migratie kan er besloten worden om alle NT4
domeinen samen te voegen tot één groot Windows Server domein. Dit kan op twee
manieren. In de eerste manier worden alle domeinen eerst samgevoegd, en
vervolgens geupgraded naar Windows Server. De oorspronkelijke domeinen worden
dan omgezet in OUs. Er zijn hiervoor geen hulpmiddelen. De migratie gebeurt
volledig manueel met behulp van commando lijn opdrachten. Als alternatief worden
de afzonderlijkde domeinen eerst geupgraded naar Windows Server. Vervolgens
wordt in elk van de masterdomeinen voor de resourcedomeinen een OU aangemaakt.
De machines in de resourcedomeinen worden dan verplaatst naar de OUs in de
masterdomeinen.

\chapter{Active Directory server rollen}

Welke vragen moet men zich stellen na de initiële installatie van een Windows
Server toestel, in veband met bijzondere functies die de server kan vervullen
met betrekking tot Active Directory? Formuleer bij het beantwoorden van deze
vragen telkens (voor zover relevant):
\begin{itemize}
	\item Hoe bepaald wordt welke servers een dergelijke specifieke functie
		vervullen?  Hoeveel zijn er nodig (in termen van:
		minimaal/exact/maximaal aantal, in functie van ...), en waarom?
	\item Eigenschappen zoals bedoeling, noodzaak, kriticiteit, inhoud,
		synchronisatie, voor welke Windows versie(s) van toepassing,...?
	\item De eventuele relatie tussen de diverse functies. Vermeld
		bijvoorbeeld welke functies al dan niet door dezelfde server
		kunnen vervuld worden, of misschien juist wel door dezelfde
		server moeten vervuld worden.
	\item Op welke diverse manieren men de toewijzing van elke bijzondere
		functie kan instellen, wijzigen en/of ongedaan maken?
\end{itemize}

\section{Wordt server al dan niet opgenomen in een domein?}

Als een Windows Server toestel lid is van een werkgroep, niet van een domein,
dan wordt het een zelfstandige server genoemd. Deze profiteren niet van de
voordelen die AD biedt. Vooral indien er reeds een AD domeinstructuur aanwezig
is, is de keuze snel gemaakt.

\section{Vervult de in een domein opgenomen server al dan niet de functie van
domeincontroller?}

Een in een domein opgenomen server die de functie van domeincontroller niet
vervult, wordt een lidserver (member server) genoemd. Meestal fungeren
lidservers als file servers, toepassingsservers, database servers, web servers,
firewalls, routers of een combinatie hiervan. In Windows Server 2008+ worden
dergelijke functies gegroepeerd op drie niveau's:
\begin{itemize}
	\item Server rollen (DNS, DHCP, ...) en Web Server implementeren
		primaire serverfuncties.
	\item Sommige, meer complexe, server rollen bieden optionele
		componenten, vertolkt door role services.
	\item Features (Powershell, SNMP, backup, ...) zorgen voor meer
		ondersteunende functies.
\end{itemize}
Men kan aanvullende server rollen, rol services en features configureren via
respectievelijk de Add Roles, Add Role Services en Add Features wizards van
Server Manager. Dit kan ook met de ServerManagerCmd opdracht.

Zowel voor lidservers als voor domeincontrollers gelden groepsbeleidinstellingen
die gedefinieerd zijn voor het domein, de OUs of de site. Lidserver behouden, in
tegestelling tot domeincontrollers, een eigen lokale beveiligingsdatabank, de
Security Account Manager (SAM).
Doorgaans configureert men slechts een factie van de Windows Server toestellen
als domeincontroller. Minimaal 1 per domein.

\section{Indien gekozen wordt voor domeincontroller, moet ook de functie van
globale catalogus ondersteund worden?}

Indien er slechts een domein in het forest is, mogen alle domeincontrollers tot
global catalog server worden gepromoveerd. Om het replicatieverkeer te beperken,
zorgt me er best voor dat er steeds een global catalog server per site aanwezig
si. De promotie gebeurt door de global catalog optie aant evinken in de general
tabpagina van de properties van de NSDS settings van een domeincontroller. Dit
is terug te vinden in de AD sites and services snap-in. De global catalog server
heeft een kopie van alle objecten van de domeingegevens van het domein waarin de
blobal catalog server zich bevindt, en een kopie van een subset van de
eigenschappen van alle objecten van het gehele forest. Deze subset wordt enkel
gerepliceerd tussen GC servers. Verder bevat de GC nog een kopie van de
configuratiegegevens van alle domeinen in het forest, en het unieke schema van
het forest. Eventueel worden ook nog specifieke applicatiepartities bijgehouden
in de global catalog.

\section{Welke domain controllers worden als RODC ingesteld?}

Domeincontrollers configureren als Read Only Domain Controller is goede praktijk
als de fysieke beveiliging van het toestel niet kan worden gegarandeerd, of waar
specifieke interactieve toepassingen enkel op een DC kunnen worden uitgevoerd.
Eeen bijkomend voordeel is dat het replicatieverkeer in een enkele richting
wordt beperkt. Er kan ook nog configuratie van een filtered attribute set (welke
kenmerken worden gerepliceerd, en welke niet) en een password replication policy
worden geconfigureerd met credential caching voor specifieke gebruikers en
computers. RODC's kunnen tevens de rol van GC en DNS server vervullen. In het
geval van DNS gaat het over een ordinaire secundaire nameserver. Een RODC biedt
ook ondersteuning voor de operations masters rollen. Er geldt wel de beperking
dat een RODC enkel met Windows Server 2008 DC's gegevens kan repliceren.

\section{Welke domeincontrollers vervullen de operations master rollen?}

\subsection{Rollen uniek in elk domein}
\begin{itemize}
	\item RID master
	\item PDC emulator
	\item infrastructure master
\end{itemize}

\subsection{Rollen uniek in het forest}
\begin{itemize}
	\item schema master
	\item domain naming master
\end{itemize}


\part{Schriftelijk}
\chapter{Active Directory replicatie}

\section{Wat is de bedoeling van replicatie?}

Gebruikers en services moeten op elk gewenst moment vanaf elke computer in het
forest toegang kunnen krijgen tot de directory gegevens. Een domein zonder
actieve domeincontroller functioneert niet langer naar gebruikers toe. Doordat
in één domein met meerdere domeincontrollers kan gewerkt worden, kan men de
fouttolerantie en de belastingsverdeling verbeteren. Bij aanpassingen (CRUD) aan
de AD gegevens moeten deze aanpassingen doorgegeven worden aan andere
domeincontrollers.

AD omvat een replicatieservice waarmee directory gegevens gedistribueerd worden
in het netwerk. Alle domeincontrollers in een domein nemen deel aan de
replicatie en bevatten een volledige kopie van alle directory gegevens voor het
eigen domein. Analoog beschikken alle domeincontrollers in een forest over het
directory schema en de configuratiegegevens, en wordt deze informatie
gedistribueerd doorheen het hele forest. De replicatieservice zorgt ervoor dat
directory gegevens altijd actueel zijn. Behalve voor wat de operations master
functies betreft, zijn alle Windows Server domeincontrollers in een domein dan
ook equivalent.

\section{Hoe wordt dit in Windows Server (ondermeer te opzichte van NT4)
gerealiseerd: Bespreek de verschillende technische kenmerken en concepten van
Windows Server replicatie, en hoe men specifieke problemen vermijdt en oplost.}

In AD wordt multi-master replicatie gebruikt, zodat de directory kan bijgewerkt
worden vanaf elke domeincontroller. Dit is een evolutie van het in NT4 gebruikte
master-slave model met primaire domeincontrollers en back-up controllers.

Het is belangrijk om ervoor te zorgen dat het replicatie verkeer relatief
beperkt blijft. In AD worden daarom alleen gewijzigde directory gegevens
gerepliceerd. Om ervoor te zorgen dat een specifieke wijziging niet meerdere
malen naar dezelfde domeincontroller wordt gerepliceerd, wordt er gebruik
gemaakt van een 64-bits Update Sequence Number (USN). Samen met het GUID van een
domeincontroller wordt dit de Up-to-Dateness Vector (UTD vector) genoemd. Elke
domeincontroller houdt in een tabel bij wat de meest recente UTD vector is die
hij van elke andere controller ontvangen heeft. Dit uitwisselingsprotocol is op
te volgen voor elke individuele partitie met behulp van de repadmin opdracht.

De metadata van elk object houdt van elk kenmerk ondermeer een Property Version
Number (PVN) bij, samen met de corresponderende UTD vector (van de
domeincontroller die de wijziging uitgevoerd heeft, op het ogenblik van de
wijziging). Op basis van de UTD vectortabel kan een controller alle wijzigingen
met een bepaalde minimale USN aan een andere controller opvragen. Aan de hand
van de metadata kan deze controller precies te weten komen welke wijzigingen hij
moet opsturen.

Met het multi-master model is het mogelijk dat hetzelfde kenmerk op meer dan één
domeincontroller tegelijkertijd gewijzigd wordt. Dit wordt opgelost door de
wijziging op het oudste tijdstip te negeren. Bij exact dezelfde tijd wordt de
wijziging van de domeincontroller met het hoogste GUID geaccepteerd. Voor
sommige objecten (schema objecten) is dit niet goed genoeg en kunnen verzoeken
tot wijzigingen enkel door 1 specifieke domeincontroller verwerkt worden.

Verwijdering van objecten biedt een bijkomend probleem: men moet vermijden dat
het object opnieuw gecreëerd wordt door replicatie vanuit een andere
domeincontroller. Een object wordt hiervoor niet onmiddelijk uit AD verwijderd,
maar als tombstone object gemarkeerd, en in een hidden container (Deleted items)
geplaatst. Pas echt verwijderd na tombstoneLifeTime (default 60 of 180) dagen.

Nog een belangrijk verschil met NT4 is de \begrip{store-and-forward} replicatie:
elke verandering op een specifieke controller wordt slechts uitgewisseld met
enkele andere domeincontrollers, die op hun beurt de wijzigingen communiceren
met nog enkele andere domeincontrollers. De gebruikte replicatietopologie wordt
automatisch gegenereerd door de Knowledge Consistency Checker (KCC) software op
elke AD domeincontroller. Via de hierbij aangemaakte verbindingsobjecten zoekt
AD zelf de meest optimale manier om het repliceren zo snel mogelijk uit te
voeren. Er worden hierbij individuele topologieën gecreëerd voor de
domeingegevens enerzijds, en voor de schema en configuratiegegevens anderzijds.
Ook voor elke applicatiepartitie is er een aparte replicatietopologie.

Windows Server replicatie is een pull, en geen push mechanisme. De
domeincontrollers brengen elkaar wel op de hoogte van de wijzigingen, maar het
opvragen gebeurt op iniatief van de replicatiepartner. Wijzigingen worden ook
opgespaard in tijdsintervallen van standaard vijf minuten en pas daarna
verstuurd. Dit wordt propagation damping genoemd en wordt uitgeschakeld voor
objecten die met beveiliging te maken hebben. Automatisch om het uur polling als
er geen meldingen van een domeincontroller zijn gekomen.

Windows Server replicatie is multi-threaded: een domeincontroller kan simultaan
repliceren met diverse partners. Een laatste verschil met NT4 is de kleinste
replicatie entiteit: in NT4 was dit een volledig object, in windows 2000
domein functioneel niveau is dit een heel attribuut en vanaf windows server 2003
functioneel niveau is de atomaire waarde van een attribuut (1 veld in
multi-valued attribuut).

\section{Welke toestellen repliceren onderling in een forest? Welke
specifieke gegevens worden hierbij uitgewisseld}

Domeincontrollers in een domein repliceren de domeingegevens van dat domein met
elkaar. Naar elke Global Catalog wordt een subset van de domeingegevens van elk
domein gerepliceerd. Het schema en configuratiegegevens worden gerepliceerd naar
alle domeincontrollers in het forest.

\section{Welke impact hebben sites met betrekking tot de replicatie van Active
Directory gegevens? Je hoeft het begrip site op zich niet verder te behandelen.}

Tussen sites is er automatisch en continu replicatie. Het is de bedoeling een
evenwicht te vinden tussen enerzijds behoufte aan actuele directory gegevens en
anderzijds beperkingen die door beschikbare netwerk bandbreedte opgelegd worden.

Inter-site replicata vertoont heel wat implementatie verschillen met intra-site
replicatie:
\begin{itemize}
	\item Enkel polling, urgent replication is bijgevolg niet mogelijk.
	\item Standaard compressie, kan uitgeschakeld worden, reduceert volume
		tot 40%
	\item Met behulp van sitekoppelingen aan te geven hoe verschillende
		sites onderling, met rechtstreekse fysieke netwerkverbindingen,
		verbonden zijn.
		\begin{itemize}
			\item KCCs genereren automatisch enkel
				verbindignsobjecten tussen sites als er tussen
				beide sites een sitekoppeling bestaat.
			\item Fouttolerantie: AD houdt zoveel mogelijk rekening
				met meerdere routes tussen sites om wijzigingen
				te repliceren.
			\item Alternatieve sitekoppeling maken de replicatie
				betrouwbaarder, als backup voor een falende
				sitekoppeling/verbinding.
		\end{itemize}
\end{itemize}

\chapter{Active Directory sites}

\section{Welke rol vervullen sites? Welke Active Directory aspecten worden
erdoor beïnvloed? Bespreek hoe deze aspecten anders gerealiseerd worden indien
de toestellen zich al dan niet in verschillende sites bevinden. (ondermeer
verschil tussen intrasite en intersite replicatie)}

Het repliceren tussen domeincontrollers kan veel bandbreedte opslorpen. In een
LAN omgeving is dit niet direct een groot probleem, maar in WAN omgevingen moet
er wel de nodige aandacht aan besteed worden. De oplossing hiervoor in Windows
Server is het gebruik van sites.

Sites komen overeen met fysieke locaties: een verzameling subnetwerken die
onderling aan LAN snelheden met elkaar kunnen communiceren. Zoals domeinen en
OUs zorgen voor een logische opsplitsing van het internetwerk, zorgen sites voor
een fysieke structurering van het netwerk. Logische en fysieke structuren zijn
onafhankelijk van elkaar.

Het gebruik van sites heeft niet enkel voordelen bij replicatie. Voor een vlotte
werking van AD is het belangrijk dat er een domeincontroller zo dicht mogelijk
bij de gebruikers geplaatst wordt. Sites maken het mogelijk om bij AD
raadplegingen eerst op zoek te gaan naar domeincontrollers die zich in dezelfde
site bevinden als de client.

Binnen een site worden directory gegevens continu en automatisch gerepliceerd.
Tussen sites moet er een evenwicht gevonden worden tussen enerzijds de behoefte
aan actuele directory gegevens en anderzijds de bandbreedte beperkingen.
Inter-site replicatie vertoont dan ook heel wat implementatie verschillen met
intra-site replicatie:
\begin{itemize}
	\item Het meldingsmechanisme wordt standaard achterwege gelaten, er
		wordt enkel gebruik gemaakt van polling. Urgent replication is
		dan ook niet mogelijk. Standaard polling om de 3 uur.
	\item Gegevensuitwisseling wordt standaard gecomprimeerd. Reduceert
		volume tot 40\%.
	\item Beheerders kunnen met behulp van sitekoppelingen aangeven hoe de
		verschillende sites onderling, met rechtstreekse fysieke
		netwerkverbindingen, verbonden zijn. De KCC's genereren
		automatisch enkel verbindingsobjecten tussen sites als er tussen
		beide sites een sitekoppeling bestaan. Een site zonder
		sitekoppelingen is zinloos, want er worden dan geen
		verbindingsobjecten aangemaakt.
\end{itemize}

Om meerdere verbindingsobjecten te vermijden die dezelfde sitekoppeling
gebruiken, wordt er automatisch (grootste GUID) per site een Inter-site Topology
Generator (ISTG) ingesteld. Enkel de KCC's van de ISTGs gebruiken de informatie
in de sitekoppelingen om verbindingsobjecten tussen controllers van
verschillende sites te genereren. De domeincontrollers die van dit uniek
verbindingsobject gebruik maken, worden bruggenhoofd (bridgehead) servers
genoemd. Een bruggenhoofd server is het punt waar directory gegevens tussen
sites uitgewisseld worden. Door de invoer van ISTGs en bridgehead servers wordt
ervoor gezorgd dat KCCs zich optimaal aanpassen aan de siteconfiguratie.

\section{Welke relaties bestaan er tussen sites, domeinen, domeincontrollers en
global catalogs? Druk deze relaties ondermeer uit in termen zoals ... een X
vereist minimaarl/exact/maximaarl Ys.... Geef een verantwoording voor elk van
deze beweringen.}

Er is geen enkel verband nodig tussen sites en domeinen. Meerdere sites in 1
domein of meerdere domeinen in 1 site zijn mogelijk.

Sites zijn enkel zinvol indien er minstens één domeincontroller in
geconfigureerd is. Als alle domeincontrollers in een site tijdelijk
onbeschikbaar zouden zijn, wordt dynamisch de dichstbijzijnde domeincontroller
van een andere site toegewezen (site covering). Alle domeincontrollers in een
site mogen RODCs zijn, Windows Server 2003 past dan wel automatisch site
covering toe, maar kan uitgeschakeld worden in het register van die controller.

Omdat het voor verificatie van gebruikers in een AD domein steeds nodig is om
vooraf een global catalog gecontacteerd te hebben, doet men er best aan om in
elke site van ten minste één domeincontroller ook een global catalog te maken.
Indien het forest slechts uit één domein bestaat, is er geen enkel bezwaar om
van alle domeincontrollers een global catalogus te maken. Dankzij caching kan
het inlogproces van een gebruiker afgewerkt worden, zonder een global catalogus
te contacteren. Hiervoor moet het forest minimaal op Windows Server 2003 forest
functioneel niveau staan.

\section{Hoe wordt bepaald tot welke site computers behoren?}

Voor werkposten wordt dit dynamisch bepaalt, telkens de IP software van een
werkpost wordt opgestart, bepaalt de werkpost via het netwerkadres in welke site
het zich bevindt.

De site locatie van een domeincontroller wordt daarentegen bepaald door de vraag
tot welke site in de directory het server object van de domeincontroller
behoort. Elke site heeft een container met de naam Servers, die alle
domeincontroller objecten bevatten die in deze site zijn geplaatst. Tijdens de
promotie van server tot domeincontroller wordt de server automatisch toegevoegd
aan de site waaraan het subnet, waartoe de server behoort, is gekoppeld. Er is
ook een Default-First-Site container voor als het subnet niet tot een bestaande
site behoort.

\section{Bespreek de diverse noodzakelijke instellingen om de verschillende
aspecten van sites te configureren, en vermeld hierbij telkens: waar en in welke
gegroepeerde vorm ze opgeslagen worden, waarom ze noodzakelijk zijn (ondermeer
welke andere aspecten er afhankelijk van zijn)}

De meeste informatie in verband met sites wordt in Active Directory zelf
bijgehouden, in de Sites container van de configuratiegegevens. Enkele default
instellingen van KCC en van het replicatie mechanisme worden echter, afhankelijk
van de Windows Server versie, niet in AD bewaard, maar in het register van elke
individuele domeincontroller.

Alle wijzigbare parameters staan gegroepeerd in de Parameters subtak van de NTDS
service. Zo houdt de parameter Repl topology update period (secs) de periode bij
van het KCC proces, en Replicator notify pause after modify (secs) de
latentieperiode voor propagation damping bij intra-site replicatie.

Meer elementaire configuratiebewerkingen met sites kunnen via de MMC snap-in:
Active Directory Sites and Services. Deze wordt aangeboden in de standaard
console sssite.msc. Wordt gebruikt om sites te creëren en om servers naar een
andere site te verplaatsen. Ook kan met behulp van site koppelingen worden
aangegeven hoe de verschillende sites onderling verbonden zijn. Welke protocol,
het (gebruiks)schema, de bandbreedte en de frequentie kan hier ingesteld worden.

\chapter{Gedeelde mappen en NTFS}

\section{Welke manier om gedeelde mappen te creeren biedt de meeste
configuratieinstellignen aan? Bespreek het doel van deze diverse instellingen en
de belangrijkste eigenschappen en mogelijkheden ervan.}

Om gedeelde mappen aan te kunnen maken, moet je over de juiste machtigingen
beschikken. Er zijn in Windows Server 2008+ diverse manieren om shares te maken:
\begin{itemize}
	\item In Explorer
	\item Shared folders snap-in
	\item Server Manager
	\item Command Prompt
\end{itemize}

De Server Manager biedt de meeste configuratie instellingen.

Je kan een nieuwe share maken door in Server Manager de Roles\textbackslash File
Services\textbackslash Share and Storage Management map te selecteren en
vervolgens de Provision Share taak uit te voeren. In de opeenvolgende dialogen
kunnen volgende zaken geconfigureerd worden:
\begin{itemize}
	\item Shared Folder Location is het pad tot de map die je wil delen
	\item NTFS Permissions
	\item SMB Permissions
	\item SMB Settings
	\item Quota Policy
	\item Filescreen Policy
	\item DFS Namespace publishing
\end{itemize}

\subsection{NTFS Permissions}

Na machtigingen op shares zijn NTFS machtigingen de tweede (beveiliging van
mappen) en derde (beveiliging van bestanden) verdedigingsmuren voor het
beveiligen van gegevens en netwerkbronnen. Er bestaan twee niveaus van NTFS
machtigingen:
\begin{itemize}
	\item Het laagste niveau bestaat uit 13 atomaire of speciale
		machtigingen, die de bouwstenen vormen voor het hogere niveau.
		Dit zijn de kleinst mogelijke machtigingen die je kan instellen.
		Bieden de mogelijkheid om zeer nauwkeuring het toegangsniveau te
		bepalen.
	\item Het hoogste niveau, moleculaire of standaard machtigingen, omvat 6
		veel gebruikte combinaties van atomaire machtigingen.
\end{itemize}

\subsection{SMB Permissions}

Share/SMB machtigingen vormen een eerste beveiligingslaag. Hebben overhand op
NTFS als ze meer beperkend zijn. Op share niveau kun je drie machtigingen
definiëren:
\begin{itemize}
	\item Full Control
	\item Read: bekijken en toepassingen uitvoeren
	\item Change: wijzigen, verwijderen. De eigenlijke share en NTFS
		machtigingen kunnen niet aangepast worden.
\end{itemize}

\subsection{SMB Setting}

\begin{description}
	\item[User limit] maximum aantal users die de share gezamelijk kunnen
		benutten
	\item[Acces based Enumeration] bestanden en submappen waar je geen
		enkele NTFS machtiging op hebt niet weergeven.
	\item[Client side caching] cachen van veel gebruikte netwerkbestanden.
\end{description}

\subsection{Quota policy}

Mogelijkheid om de hoeveelheid beschikbare ruimte voor gebruikers in te stellen.
\begin{description}
	\item[Volumequota] Gebaseerd op bestandseigendom en dus onmogelijk om
		per gebruikersgroep de som te laten controleren.
	\item[Mapquota] Vanaf Windows Server 2008 kan men wel mapquota op mappen
		plaatsen, maar dit geldt voor alle gebruikers. Werkt met soft en
		hard treshold.
\end{description}

\subsection{Filescreen Policy}

Mogelijkheid tot verhindern van opslaan van bestanden met een bepaalde extensie.
Er kunnen reacties getriggerd worden als iemand dit toch probeert.

\subsection{DFS Namespace publishing}

Distributed File System namespace is equivalent en fungeert zoals een share, met
het verschil dat een DFS naespace kan bestaan uit DFS mappen (die verwijst naar
een share, niet noodzakelijk op dezelfde server of zelf een DFS namespace),
bestanden en gewone mappen. Dit geeft de mogelijkheid om transparant aan
gebruikers bronnen van meerdere servers op 1 locatie aan te bieden. De DFS
topologie wordt automatisch in AD gepubliceerd en is dus altijd zichtbaar voor
gebruikers op alle servers in het domein.

DFS folder targets geven de mogelijkheid om identieke shares samen te nemen tot
1 DFS map. Een client zal dan bij het connecteren tot een DFS namespace elke
folder target uitproberen tot een werkende gevonden wordt. Dit geeft een hogere
fouttolerantie en een spreiding van de belasting. Synchronisatie van deze
verschillende shares gaat met FRS of DFS-R.

\section{Waar wordt de definitie en (partiële) configuratie van gedeelde mappen
opgeslagen? Hoe kan men deze wijzigen vanuit de command prompt?}

De configuratie wordt opgeslagen in het register van de server. Per share 1
multi-valued sleutel in de Shares subtak van de LanManServer service.

\begin{description}
	\item[net file] Bekijk en beheer bronnen die gedeeld worden over het
		netwerk. De server service moet draaien om dit commando te
		kunnen gebruiken.
	\item[net config] Bekijk het maximum aantal gebruikers die een gedeelde
		bron kunnen raadplegen en de maximum aantal open files per
		sessie.
	\item[net use] Verbind of verbreek de verbinding met een computer van
		een gedeelde bron.
	\item[net session] Beheer server connecties.
	\item[net share] Maak, verwijder, beheer en toon gedeelde bronnen.
	\item[net view] Toon informatie over de domeinen, computers en bronnen
		die gedeeld zijn door de aangegeven computer, inclusief de
		offline client cache setting.
	\item[net help] Bekijk de hulp pagina voor de netwerk commando's.

\end{description}

\section{Op welke diverse manieren kan men gebruik maken van gedeelde mappen?}

Eenvoudigste manier: rechtstreeks pad ingeven in de adresbalk van een Explorer.
Als je een bepaalde netwerklocatie frequent gebruikt, is het aangewezen om deze
locatie in het lokale filesysteem te mappen. De share krijgt dan een
stationsletter toegewezen, en kan net als een lokaal station gebruikt worden.
Dit is in te stellen door in explorer rechtermuisknop op Network of My network
places, dan Map Netwrok Drive en klik op wizard. Kan ook met het command net
use.

\section{Geef een overzicht van de belangrijkste voordelen van de opeenvolgende
versies van het NTFS-bestandssysteem. Bespreek elk van deze aspecten (ondermeer
het doel, de voordelen en de beperkingen ervan), en geef aan hoe je er gebruik
kan van maken, bij voorkeur vanuit een Command Prompt.}

\subsection{Features vanaf v1.2}

\begin{itemize}
	\item Beveiliging op bestandsniveau
	\item Logging van schijfactiviteiten
	\item Dynamisch uitbreiden van partities/volumes
	\item Compressie
	\item Grotere partities, zonder performantiedegradatie
	\item Hardlinks
	\item Auditing op objecttoegang
\end{itemize}

\subsection{Features vanaf v3.0}

\begin{itemize}
	\item Reparsepunten en bestandssysteemfilters
	\item Transparante encryptie en decodering
	\item Individuele diskquota op volumeniveau
	\item Volumekoppelpunten: mounten van volumes in NTFS mappen
	\item Sparse bestanden
	\item Junction points naar mappen (soft link)
\end{itemize}

\chapter{Machtigingen op bestandstoegang}

\section{Welke rol spelen machtigingen bij de beveiliging van bronnen? Geef een
gedetaileerd algemeen overzicht van het mechanisme van machtigingen.}

Met machtigingen bepaal je wie toegang heeft tot welke gegevens, en wat die er
mee kan doen. Elk object in AD, elk object van een NTFS volume, elke
registersleutel, elk proces en ook elke service heeft een security descriptor.
Deze bevat ondermeer:
\begin{itemize}
	\item Een machtigingsset of Discretionary Acces Control List (DACL).
	\item De System Access Control List (SACL) definieert welke acties van
		de gebruiker gelogd worden.
	\item De SID van de eigenaar van het object. Makers van objecten zijn
		standaard eigenaar, maar de eigendom kan worden overgedragen. De
		eigenaar of beheerder is verantwoordelijk voor het instellen van
		de ACL.
	\item De primaire group van de maker, enkel van belang voor POSIX
		toepassingen.
\end{itemize}

De ACL is een verzameling machtigingen die bepaalt welke gebruiker of welke
groep welke toegangsrechten heeft voor het object. De specifieke machtigingen
die kunnen worden toegekend is afhankelijk van het object waarop ze worden
toegepast.

De ACL bestaat uit Access Control Entries (ACEs) of machtigingsvermeldingen. Een
ACE kan zowel een machtiging toekennen als ontzeggen. Machtigingen worden best
zo veel mogelijk op groepen toegepast om beheer te vereenvoudigen.

ACEs worden in canonieke volgorde verwerkt. Eerst komen de ACEs die machtigingen
ontzeggen aan de beurt, daarna degene die toekennen. Ontzeggen is steeds het
sterkste kenmerk, een toekenning van een machtiging wordt genegeerd als die
eerder ontzegd werd. Afwijken van deze volgorde kan enkel programmatorisch.

Machtigingen zijn cumulatief. De machtigingen van een groep worden op zijn leden
toegepast, tenzij die machtiging elders (op de specifieke gebruiker of een
andere groep) ontzegd werden. Machtigingen worden impliciet geweigerd.

Er zijn twee soorten machtigingen:
\begin{description}
	\item[Expliciete machtigingen] rechtstreeks aan het object gekoppeld.
	\item[Overgenomen machtigingen] machtingen overgenomen van de container
		waartoe het object behoort. Dit vereenvoudigt het beheer van
		machtigingen sterk.
\end{description}

Expliciete machtigingen krijgen altijd voorrang op impliciete. Het ontbreken van
een ACL is een ernstig risico. Immers objecten zonder ACL zijn voor iedereen
toegankelijk, terwijl een lege ACL ervoor zorgt dat toegang impliciet wordt
geweigerd.

\section{Bespreek hoe het mechanisme van machtigingen specifiek (en op diverse
niveaus) toegepast wordt op bestandstoegang. Geef de verschillende soorten
machtigingen, hun onderlinge relaties, en hoe deze kunnen geanalyseerd en
ingesteld worden. Toon hierbij aan dat je zelf met deze configuratietools
geëxperimenteerd hebt.}

Er zijn twee niveau's van machtigingen voor bestandstoegang: Share/SMB
machtigingen en NTFS machtigingen.

Share machtigingen krijgen de bovenhand als ze beperkender zijn dan de NTFS. Men
zou dus de share machtigingen op FULL Control kunnen zetten voor iedereen en
verdere vernauwing van de rechten aan de hand van NTFS machtigingen bekomen, dit
heeft echter 2 problemen:
\begin{enumerate}
	\item FAT bestandssystemen kennen geen NTFS machtigingen.
	\item Gebruikers met volledige share machtigingen kunnen alle submappen
		zien ook al hebben ze 0 NTFS machtigingen hierop. Dit kan
		opgelost worden met Access Based Enumeration.
\end{enumerate}

\subsection{SMB machtigingen}

Dit is de 1e beveiligingsmuur voor bestandstoegang. Ze kunnen op drie manieren
ingesteld worden: in Explorer, in het detailpaneel van de Shares map van
compmgmt.msc en in het detailpaneel van de Share and Storage Management map van
Server Manager. Er zijn 3 machtigingen, instelbaar per gebruiker of groep:
\begin{description}
	\item[Full Control] Alle bewerkingen op alle bestanden/mappen. Ook de
		share zelf kan worden gewijzigd.
	\item[Read] Kan de hele hiërarchie van de share bekijken, elke file
		lezen en toepassingen uitvoeren.
	\item[Change] Bovenop alles van Read ook wijzigen en verwijderen. Ook
		bestandskenmerken. Share zelf en NTFS machtigingen kunnen niet
		gewijzigd worden.
\end{description}

\subsection{NTFS machtigingen}

Na machtigingen op shares zijn NTFS machtigingen de tweede (beveiliging van
mappen) en derde (beveiliging van bestanden) verdedigingsmuren voor het
beveiligen van gegevens en netwerkbronnen. Er bestaan twee niveaus van NTFS
machtigingen:
\begin{itemize}
	\item Het laagste niveau bestaat uit 13 atomaire of speciale
		machtigingen, die de bouwstenen vormen voor het hogere niveau.
		Dit zijn de kleinst mogelijke machtigingen die je kan instellen 
		en bieden de mogelijkheid om zeer nauwkeuring het toegangsniveau te
		bepalen.
	\item Het hoogste niveau, moleculaire of standaard machtigingen, omvat 6
		veel gebruikte combinaties van atomaire machtigingen.
\end{itemize}

\subsubsection{Instellen van NTFS machtigingen}

NTFS machtigingen toewijzen is niet moeilijk, maar in een grote omgeving met
veel bronnen en gebruikers moet je zeer methodisch en consistent te werk gaan.
De ACL editor, van een bestand of map, kan opgeroepen worden via de Security
tabpagina van het object. Hier kunnen ACEs toegevoegd en verwijderd worden. De
grijze selectievakjes duiden overgenomen machtigingen aan.

Deze inheritance of propagation is zeer handig om op een eenvoudige manier de
toegang wil definiëren tot de volledige naamruimte van een grote maphiërarchie.

De advanced knop in Security tabpagina geeft toegang tot een volgende niveau
bestaande uit 4 tabpagina's:
\begin{enumerate}
	\item Permissions: Toont steeds de correcte weergave v/d werkelijke
		ACLs en laat aanvullende mogelijkheden toe: Allow inheritance
		permissions, Replace all existing inheritance permissions,
		Add/Edit/Delete ACEs.
	\item Auditing: Laat toe om SACL van het object in te stellen.
	\item Owner: Laat toe om het ownership over te nemen. Het eigenaarschap
		kan niet eenzijdig worden overgedragen.
	\item Effective permissions: Biedt de voor probleemdiagnose interessante 
		mogelijk om voor een specifieke gebruiker het zelfde algoritme toe te 
		passen dat de SRM gebruikt om de uiteindelijke machtigingen op een 
		object na te gaan.
\end{enumerate}

\section{Wat gebeurt er met de machtigingen bij het verplaatsen van een bestand?
Wat gebeurt er met de machtigingen bij het kopiëren van een bestand?}

De gebruiker die de actie onderneemt wordt eigenaar van de bestanden wanneer ze
bij de bestemming aankomen.

Wanneer de bestemming een container is op een ander NTFS volume, of de bestanden
met standaard tools worden gekopieerd, vervallen de expliciete machtigingen. De
machtigingen van de doelcontainer worden overgenomen door het object zelf en al
zijn onderliggende objecten. Objecten die naar een niet NTFS volume worden
gekopieerd, verliezen alle machtigingen.

Om de machtigingen tijdens een kopieeropdracht te behouden moeten speciale tools
zoals robocopy en scopy gebruikt worden.

Wanneer bestanden of mappen worden verplaatst naar een container binnen
hetzelfde volume, worden de expliciete machtigingen behouden, en worden de
machtigingen van de container overgenomen.

\section{Op welke andere objecten zijn machtigingen van toepassing?}

Elk object in AD, elk object van een NTFS volume, elke registersleutel, elk
proces, en elke service heeft een security descriptor, en dus de bijhorende
machtigingen.

Het individueel instellen van de machtigingen van elk object zou heel wat werk
veroorzaken. Gelukkig worden de meeste objecten op één of andere manier
hiërarchisch gestructureerd en ken er gebruik gemaakt worden van overerving van
machtigingen.

\chapter{Gebruikersgroepen}

\section{Bespreek in detail het onderscheid tussen de diverse soorten
veiligheidsgroepen, ondermeer afhankelijk of het toestel al dan niet in een
domein is opgenomen. Behandel hierbij vooral de mogelijkheden en beperkingen.}

Bespreek ondermeer:
\begin{itemize}
	\item De zichtbaarheid van de diverse soorten groepen
	\item Welke objecten er van kunnen zijn
	\item De onderlinge relaties en de regels voor het nesten van de diverse
		soorten groepen? Stel deze relaties eveneens schematisch voor.
\end{itemize}

Er zijn drie soorten veiligheidsgroepen, bereiken of scopes genoemd: lokaal,
globaal en universeel. Zowel het type als de scope van een groep worden
bijgehouden in de individuele bits van het groepType attribuut van het groep
object. Het bereik van een groep bepaalt zowel of een groep leden uit andere
domeinen en forests kan hebben, als de domeinen waarin rechten en machtigingen
aan de groep kunnen toegewezen worden.

\subsection{Lokale veiligheidsgroepen}

Lokale groepen kunnen leden uit elk domein van het forest of andere trusted
domeinen bevatten. Lokale groepen zijn enkel zichtbaar en geldig in het eigen
domein. Lokale groepen worden niet gekopieerd naar de global catalog. Ze zijn
niet enkel zichtbaar voor domeincontrollers maar ook op alle werkposten en
lidservers van het domein.

Lokale groepen worden typisch gebruikt om rechten en machtigingen toe te kennen,
en bevatten eerder andere groepen dan gebruikers.

\subsection{Globale veiligheisgroepen}

Globale groepen kunnen alleen gebruikers en andere globale groepen uit hetzelfde
domein omvatten. Ze zijn zichtbaar in elk domein van het forest of andere
trusting domeinen. Globale groepen kunnen ook toegelaten worden tot de lokale
groepen van werkposten en lidservers. De naam van de groep wordt gekopieerd naar
de global catalog, maar niet de leden ervan.

Ze worden vooral gebruikt als container voor gebruikers die dezelfde
machtigingen of rechten nodig zullen hebben.

\subsection{Universele veiligheidsgroepen}

Universele groepen zijn nieuw voor Windows Server en zijn enkel te gebruiken in
domeinen met minstens Windows 2000 native functioneel niveau. Ze kunnen leden
uit elk domein van het forest (niet van andere trusting domeinen) bevatten en
zijn zichtbaar in elk domein van het forest (niet in andere trusting domeinen).

Universele groepen worden net als lokale groepen typisch gebrukt om rechten en
machtigingen toe te kennen, maar bieden het voordeel dat ze in alle domeinen
tegelijkertijd geldig zijn. Ze zijn belastend voor de global catalog. Bij
voorkeur bevatten universele groepen dan ook enkel globale groepen.

\section{Hoe en waarom worden deze soorten groepen in de praktijk best gebruikt,
al dan niet gecombineerd? Van welke omstandigheden is dit afhankelijk?
Illustreer aan de hand van concrete voorbeelden.}

\begin{description}
	\item[Lokale groepen] worden typisch gebruikt om rechten en machtigingen
		toe te kennen, door voor elke bron of verzameling bronnen één of
		meerdere lokale groepen te creëren en de toegang tot die bron in
		te stellend oor één keer toegang te verlenen aan de lokale
		groepen. De toegang tot de bron wordt nadien enkel gewijzigd
		door het lidmaatschap te manipluleren. Lokale groepen zijn niet
		alleen zichtbaar op domeincontrollers, maar ook op alle
		werkposten en lidservers. Dit neemt de noodzaak weg om uit
		veiligheidsoverwegingen alle lidservers te configureren als
		domeincontrollers.
	\item[Globale groepen] worden eerder gebruikt als containers voor
		gebruikers die dezelfde machtigigen of rechten nodig zullen
		hebben en als leden van andere groepen, in het domein zelf of in
		een trusting domein. Als je een verzameling gebruikers,
		afkomstig uit een ander domein toegang wil verlenen tot een
		bepaalde bron, dan kun je deze verzameling best groeperen in een
		glogale groep (van het vreemde domein) en deze lid maken van een
		lokale groep, gekoppeld aan de bron.
	\item[Universele groepen] verenigt op het eerste zicht de beste
		karakteristieken van zowel lokale als globale groepen.
		Universele groepen worden net als lokale groepen typisch
		gebruikt om rechten en machtigingen toe te kennen, maar bieden
		het voordeel dat ze in alle domeinen tegelijkertijd geldig zijn
		en dus slechts eenmaal moeten gedefinieerd worden. Zowel de
		namen als de leden van universele groepen worden in de global
		catalog opgenomen, en dus wordt deze zeer groot. Bij voorkeur
		bevatten universele groepen enkel globale groepen.
\end{description}

\section{Welke conversieregels gelden er tussen de diverse soorten groepen.
Behandel hierbij alle mogelijke combinaties.}

Lokale groepen mogen niet naar universele groepen geconverteerd worden als die
lokale groepen andere lokale groepen bevatten. Dit omdat een universele groep
geen lokale groep kan bevatten.

Als een globale groep een andere globale groep bevat kan dat lid niet
geconverteerd worden naar een universele groep. Dit omdat globale groepen geen
universele groepen kunnen bevatten.

Er mogen verschuivingen optreden als er geen regels gebroken worden.

Universele groepen kunnen terug naar zowel globale groepen als lokale groepen
teruggebracht worden op voorwaarde dat een universele groep leden heeft die tot
1 domein behoren.

Een globale groep wordt per domein gedefinieerd integenstelling tot een
universele groep dat in de global catalog opgeslagen wordt en dus over de
verschillende domeinen van het forest gekend is. Dus als een universele groep
leden bevat over meerdere domeinen, is deze conversie niet mogelijk.

\chapter{Configuratie van gebruikersgroepen}

\section{Waar en hoe wordt het (volledige) lidmaatschap van een object tot een
groep bijgehouden? Op welke diverse manieren kan men dit lidmaatschap
configureren? Op welke diverse manieren kan men de volledige verzameling van
objecten, die er deel van uitmaken, achterhalen?}

In het properties venster van een gebruikersaccount kan men in het Member Of
tabblad lidmaatschap tot groepen configureren. Het memberOf attribuut dat
hiervoor gebruikt wordt, is gelinkt aan het member attribuut van de groep en is
hier de back-link. Het kan dus niet rechtstreeks gewijzigd worden, dit kan enkel
via het member attribuut van de groep.

In het properties venster van een groep kan men in het tabblad Members de groep
bevolken. In het Member Of tabblad kan men de groep zelf deel laten uitmaken van
een andere groep. De lijst van members wordt dus steeds opgeslagen in een
link-kenmerk met in de groep de forward-link members. En in het object dat deel
uitmaakt van de groep de back-link memberOf. Doordat memberOf een back-link
kenmerk is kan het niet rechtstreeks gewijzigd worden.

Groepsleden kunnen ook uit trusted domeinen geselecteerd worden. AD maakt
hiervoor een phantom object aan, dat het object uit het vertrouwde domein
vertegenwoordigt. Deze mapobjecten komen terecht in de container
ForeignSecurityPrincipals, en kunnen lid worden van lokale groepen in het
domein.

Uiteraard zijn de groepen ook te beheren via de Command Prompt door middel van
de opdrachten: net group, net localgroup, dsadd group, dsrm group, dsget group
en dsmod group.

De verzameling objecten die tot een groep behoren kan men verkrijgen met:
\begin{itemize}
	\item Door met het commando dsget group het attribuut members op te
		vragen van een groep.
	\item In het Members tabblad van de properties van de groep.
	\item Met behulp van een LDAP query.
\end{itemize}

\section{Door wie wordt het lidmaatschap van de diverse groepen bij voorkeur
ingesteld?}

Dit kan best door de beheerder gebeuren. Lidmaadschap bij een groep kan bepaalde
machtigingen met zich meebrengen en het is de taak van de beheerder om ervoor te
zorgen dat gebruikers niet meer rechten hebben dan dat ze nodig hebben.

In grote organisaties kan een deel van de beheerstaken gedelegeerd worden naar
andere personen. Zij zullen dan bv rechten hebben om een bepaalde subtak of OU
te beheren.

Bij subscriptiegroepen kunnen gebruikers zich zelf in en uitschrijven.

\section{Op welke diverse manieren kan men het beheer van van Active Directory
objecten, specifieke attributen van groepsobjecten in het bijzonder, delegeren
aan niet-administrators? Bespreek een aantal technieken om dit delegeren zo
eenvoudig mogelijk uit te voeren.}

Men kan in AD de beheerstaken van een OU delegeren aan een groep gebruikers.
Hiervoor moet men de ACLs van de OU instellen. Dit kan op drie manieren:
\begin{itemize}
	\item Via de Delegation of Control Wizard
	\item Via de properties van de OU
	\item Via de Command Prompt
\end{itemize}

\subsection{Delegation of Control Wizard}

Om de wizard op te starten klik je in dsa.msc met de rechtmuisknop op een OU en
selecteer je Delegate Control. Je selecteert de groepen of gebruikers waaraan je
beheersmachtigingen wil delegeren. Deze hoeven geen deel uit te maken van de
groep zelf. Het volgende dialoogvenster toont een lijst met de meest voorkomende
beheertaken die voor delegeren in aanmerking komen. Met create a custom task to
delegate kan je zelf een gedetailleerde keuze maken tussen alle beschikbare
delegeeropties. Het laatste dialoog venster toont ter bevestiging een overzicht
van de geselecteerde instellingen.

\subsection{OU properties}

Rechterklik op de OU, dan properties en naar het security tabblad gaan. Daar kan
je zowel taken delegeren als de instellingen voor reeds gedelegeerde taken
wijzigen. De lijst van molucaire machtigingen waaruit je kan kiezen is
afhankelijk van de objectklasse.

\subsection{Command Prompt}

Met de acldiag en dsacls opdrachten. Terug intrekken kan met dsrevoke.

\section{Aan welke groepen/entiteiten worden rechten in de praktijk toegekend?
Bespreek de bijzonderheden van dergelijke groepen/entiteiten, en vermeld er de
meest interessante voorbeelden van (telkens met hun bedoeling en hun
randeffecten).}

Rechten kunnen afzonderlijk aan een individuele gebruiker toegekend worden, maar
voor het organiseren van de beveiliging is het beter de gebruiker in een groep
te plaatsen, en te definieren welke rechten aan de groep toegekend worden.
Verschillende praktische groepen zijn oa:

\begin{description}
	\item[Backup Operators] hebben de bevoegdheid om bestanden en mappen te
		back-uppen en terug te plaatsen, zelfs als ze geen toestemming
		hebben om deze bestanden te lezen of te wijzigen.
	\item[Account Operator] maken en beheren gebruikersaccounts en groepen,
		en kunnen computers aan het domein toevoegen.
	\item[Server Operators] kunnen onder andere het systeem vanop afstand
		uitzetten, de systeemtijd veranderen, de harde schijf
		formatteren en mappen delen.
	\item[Print Operators] kunnen printers delen, wissen en beheren.
	\item[Administrators] hebben bijna elk recht. Toch krijgen beheerders
		met de ruime bevoegdheden van deze groep geen toegang tot alle
		bestanden en mappen.
	\item[User] kunnen programmas gebruiken, maar ze niet installeren.
\end{description}

AD kent ook impliciete groepen, deze hebben geen specifiek lidmaatschap en
kunnen op verschillende tijdstippen verschillende gebruikers vertegenwoordigen.
Enkele voorbeelden:
\begin{description}
	\item[Interactive] iedereen die de computer lokaal gebruikt.
	\item[Network] alle gebruikers die via het netwerk zijn verbonden met
		een computer.
	\item[Everyone] combinatie van de Interactive en Network groepen.
	\item[Authenticated Users] alle gebruikers die geauthentificeerd zijn.
	\item[System] het besturingssysteem, heeft bijna alle bevoegdheden.
		Processen van het OS hebben deze rechten nodig om te kunnen
		draaien.
\end{description}

\chapter{Gebruikersprofielen}

\section{Wat is de bedoeling van gebruikersprofielen? Bespreek zowel uit het
standpunt van gebruikers, als uit het standpunt van beheerders, en bespreek
hierbij de voor- en nadelen.}

\section{Geef de verschillende types gebruikersprofielen. Hoe worden deze
ingesteld, en waar worden deze bij voorkeur opgeslagen?}

\section{Geef de verschillende componenten van gebruikersprofielen.}

\section{Over welke alternatieve hulpmiddelen beschikt een beheerder om
gebruikersprofielen te configureren?}


\backmatter


\end{document}

