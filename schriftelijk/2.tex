\chapter{Active Directory sites}

\section{Welke rol vervullen sites? Welke Active Directory aspecten worden
erdoor beïnvloed? Bespreek hoe deze aspecten anders gerealiseerd worden indien
de toestellen zich al dan niet in verschillende sites bevinden. (ondermeer
verschil tussen intrasite en intersite replicatie)}

Het repliceren tussen domeincontrollers kan veel bandbreedte opslorpen. In een
LAN omgeving is dit niet direct een groot probleem, maar in WAN omgevingen moet
er wel de nodige aandacht aan besteed worden. De oplossing hiervoor in Windows
Server is het gebruik van sites.

Sites komen overeen met fysieke locaties: een verzameling subnetwerken die
onderling aan LAN snelheden met elkaar kunnen communiceren. Zoals domeinen en
OUs zorgen voor een logische opsplitsing van het internetwerk, zorgen sites voor
een fysieke structurering van het netwerk. Logische en fysieke structuren zijn
onafhankelijk van elkaar.

Het gebruik van sites heeft niet enkel voordelen bij replicatie. Voor een vlotte
werking van AD is het belangrijk dat er een domeincontroller zo dicht mogelijk
bij de gebruikers geplaatst wordt. Sites maken het mogelijk om bij AD
raadplegingen eerst op zoek te gaan naar domeincontrollers die zich in dezelfde
site bevinden als de client.

Binnen een site worden directory gegevens continu en automatisch gerepliceerd.
Tussen sites moet er een evenwicht gevonden worden tussen enerzijds de behoefte
aan actuele directory gegevens en anderzijds de bandbreedte beperkingen.
Inter-site replicatie vertoont dan ook heel wat implementatie verschillen met
intra-site replicatie:
\begin{itemize}
	\item Het meldingsmechanisme wordt standaard achterwege gelaten, er
		wordt enkel gebruik gemaakt van polling. Urgent replication is
		dan ook niet mogelijk. Standaard polling om de 3 uur.
	\item Gegevensuitwisseling wordt standaard gecomprimeerd. Reduceert
		volume tot 40\%.
	\item Beheerders kunnen met behulp van sitekoppelingen aangeven hoe de
		verschillende sites onderling, met rechtstreekse fysieke
		netwerkverbindingen, verbonden zijn. De KCC's genereren
		automatisch enkel verbindingsobjecten tussen sites als er tussen
		beide sites een sitekoppeling bestaan. Een site zonder
		sitekoppelingen is zinloos, want er worden dan geen
		verbindingsobjecten aangemaakt.
\end{itemize}

Om meerdere verbindingsobjecten te vermijden die dezelfde sitekoppeling
gebruiken, wordt er automatisch (grootste GUID) per site een Inter-site Topology
Generator (ISTG) ingesteld. Enkel de KCC's van de ISTGs gebruiken de informatie
in de sitekoppelingen om verbindingsobjecten tussen controllers van
verschillende sites te genereren. De domeincontrollers die van dit uniek
verbindingsobject gebruik maken, worden bruggenhoofd (bridgehead) servers
genoemd. Een bruggenhoofd server is het punt waar directory gegevens tussen
sites uitgewisseld worden. Door de invoer van ISTGs en bridgehead servers wordt
ervoor gezorgd dat KCCs zich optimaal aanpassen aan de siteconfiguratie.

\section{Welke relaties bestaan er tussen sites, domeinen, domeincontrollers en
global catalogs? Druk deze relaties ondermeer uit in termen zoals ... een X
vereist minimaarl/exact/maximaarl Ys.... Geef een verantwoording voor elk van
deze beweringen.}

Er is geen enkel verband nodig tussen sites en domeinen. Meerdere sites in 1
domein of meerdere domeinen in 1 site zijn mogelijk.

Sites zijn enkel zinvol indien er minstens één domeincontroller in
geconfigureerd is. Als alle domeincontrollers in een site tijdelijk
onbeschikbaar zouden zijn, wordt dynamisch de dichstbijzijnde domeincontroller
van een andere site toegewezen (site covering). Alle domeincontrollers in een
site mogen RODCs zijn, Windows Server 2003 past dan wel automatisch site
covering toe, maar kan uitgeschakeld worden in het register van die controller.

Omdat het voor verificatie van gebruikers in een AD domein steeds nodig is om
vooraf een global catalog gecontacteerd te hebben, doet men er best aan om in
elke site van ten minste één domeincontroller ook een global catalog te maken.
Indien het forest slechts uit één domein bestaat, is er geen enkel bezwaar om
van alle domeincontrollers een global catalogus te maken. Dankzij caching kan
het inlogproces van een gebruiker afgewerkt worden, zonder een global catalogus
te contacteren. Hiervoor moet het forest minimaal op Windows Server 2003 forest
functioneel niveau staan.

\section{Hoe wordt bepaald tot welke site computers behoren?}

Voor werkposten wordt dit dynamisch bepaalt, telkens de IP software van een
werkpost wordt opgestart, bepaalt de werkpost via het netwerkadres in welke site
het zich bevindt.

De site locatie van een domeincontroller wordt daarentegen bepaald door de vraag
tot welke site in de directory het server object van de domeincontroller
behoort. Elke site heeft een container met de naam Servers, die alle
domeincontroller objecten bevatten die in deze site zijn geplaatst. Tijdens de
promotie van server tot domeincontroller wordt de server automatisch toegevoegd
aan de site waaraan het subnet, waartoe de server behoort, is gekoppeld. Er is
ook een Default-First-Site container voor als het subnet niet tot een bestaande
site behoort.

\section{Bespreek de diverse noodzakelijke instellingen om de verschillende
aspecten van sites te configureren, en vermeld hierbij telkens: waar en in welke
gegroepeerde vorm ze opgeslagen worden, waarom ze noodzakelijk zijn (ondermeer
welke andere aspecten er afhankelijk van zijn)}

De meeste informatie in verband met sites wordt in Active Directory zelf
bijgehouden, in de Sites container van de configuratiegegevens. Enkele default
instellingen van KCC en van het replicatie mechanisme worden echter, afhankelijk
van de Windows Server versie, niet in AD bewaard, maar in het register van elke
individuele domeincontroller.

Alle wijzigbare parameters staan gegroepeerd in de Parameters subtak van de NTDS
service. Zo houdt de parameter Repl topology update period (secs) de periode bij
van het KCC proces, en Replicator notify pause after modify (secs) de
latentieperiode voor propagation damping bij intra-site replicatie.

Meer elementaire configuratiebewerkingen met sites kunnen via de MMC snap-in:
Active Directory Sites and Services. Deze wordt aangeboden in de standaard
console sssite.msc. Wordt gebruikt om sites te creëren en om servers naar een
andere site te verplaatsen. Ook kan met behulp van site koppelingen worden
aangegeven hoe de verschillende sites onderling verbonden zijn. Welke protocol,
het (gebruiks)schema, de bandbreedte en de frequentie kan hier ingesteld worden.
