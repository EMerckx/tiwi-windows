\chapter{Machtigingen op bestandstoegang}

\section{Welke rol spelen machtigingen bij de beveiliging van bronnen? Geef een
gedetaileerd algemeen overzciht van het mechanisme van machtigingen.}

Met machtigingen bepaal je wie toegang heeft tot welke gegevens, en wat die er
mee kan doen. Elk object in AD, elk object van een NTFS volume, elke
registersleutel, elk proces en ook elke service heeft een security descriptor.
Deze bevat ondermeer:
\begin{itemize}
	\item Een machtigingsset of Discretionary Acces Control List (DACL).
	\item De System Access Control List (SACL) definieert welke acties van
		de gebruiker gelogd worden.
	\item De SID van de eigenaar van het object. Makers van objecten zijn
		standaard egenaar, maar de eigendom kan worden overgedragen. De
		eigenaar of beheerder is verantwoordelijk voor het instellen van
		de ACL.
	\item De primaire group van de maker, enkel van belang voor POSIX
		toepassingen.
\end{itemize}

De ACL is een verzameling machtigingen die bepaalt welke gebruiker of welke
groep welke toegangsrechten heeft voor het object. De specifieke machtigingen
die kunnen worden toegekend is afhankelijk van het object waarop ze worden
toegepast.

De ACL bestaat uit Access Control Entries (ACEs) of machtigingsvermeldingen. Een
ACE kan zowel een machtiging toekennen als ontzeggen. Machtigingen worden best
zo veel mogelijk op groepen toegepast om beheer te vereenvoudigen.

ACEs worden in canonieke volgorde verwerkt. Eerst komen de ACEs die machtigingen
ontzeggen aan de beurt, daarna degene die toekennen. Ontzeggen is steeds het
sterkste kenmerk, een toekenning van een machtiging wordt genegeerd als die
eerder ontzegt werd. Afwijken van deze volgorde kan enkel programmatorisch.

Machtigingen zijn cumulatief. De machtigingen van een groep worden op zijn leden
toegepast, tenzijn die machtiging elders (op de specifieke gebruiker of een
andere groep) ontzegd werden. Machtigingen worden impliciet geweigerd.

Er zijn twee zoorden machtigingen:
\begin{description}
	\item[Expliciete machtigingen] rechtstreeks aan het object gekoppeld.
	\item[Overgenomen machtigingen] machtingen overgenomen van de container
		waartoe het object behoort. Dit vereenvoudigt het beheer van
		machtigingen sterk.
\end{description}

Expliciete machtigingen krijgen altijd voorrang op impliciete. Het ontbreken van
een ACL is een ernstig risico. Immers objecten zonder ACL zijn voor iedereen
toegankelijk, terwijl een lege ACL ervoor zordt dat toegang impliciet wordt
geweigerd.

\section{Bespreek hoe het mechanisme van machtigingen specifiek (en op diverse
niveaus) toegepast wordt op bestandstoegang. Geef de verschillende soorten
machtigingen, hun onderlinge relaties, en hoe deze kunnen geanalyseerd en
ingesteld worden. Toon hierbij aan dat je zelf met deze configuratietools
geëxperimenteerd hebt.}

Er zijn twee niveau's van machtigingen voor bestandstoegang: Share/SMB
machtigingen en NTFS machtigingen.

Share machtigingen krijgen de bovenhand als ze beperkender zijn dan de NTFS. Men
zou dus de share machtigingen of FULL Control kunnen zetten voor iedereen en
verdere vernauwing van de rechten aan de hand van NTFS machtigingen bekomen, dit
heeft echter 2 problemen:
\begin{enumerate}
	\item FAT bestandssystemen kennen geen NTFS machtigingen.
	\item Gebruikers met volledige share machtigingen kunnen alle submappen
		zien ook al hebben ze 0 NTFS machtigingen hierop. Dit kan
		opgelost worden met Access Based Enumeration.
\end{enumerate}

\subsection{SMB machtigingen}

Dit is de 1e beveiligingsmuur voor bestandstoegang. Ze kunnen op drie manieren
ingesteld worden: in Explorer, in het detailpaneel van de Shares map van
compmgmt.msc en in het detailpaneel van de Share and Storage Management map van
Server Manager. Er zijn 3 machtigingen, instelbaar per gebruiker of groep:
\begin{description}
	\item[Full Control] Alle bewerkingen op alle bestanden/mappen. Ook de
		share zelf kan worden gewijzigd.
	\item[Read] Kan de hele hiërarchie van de share bekijken, elke file
		lezen en toepassingen uitvoeren.
	\item[Change] Bovenop alles van Read ook wijzigen en verwijderen. Ook
		bestandskenmerken. Share zelf en NTFS machtigingen kunnen niet
		gewijzigd worden.
\end{description}

\subsection{NTFS machtigingen}

Na machtigingen op shares zijn NTFS machtigingen de tweede (beveiliging van
mappen) en derde (beveiliging van bestanden) verdedigingsmuren voor het
beveiligen van gegevens en netwerkbronnen. Er bestaan twee niveaus van NTFS
machtigingen:
\begin{itemize}
	\item Het laagste niveau bestaat uit 13 atomaire of speciale
		machtigingen, die de bouwstenen vormen voro het hogere niveau.
		Dit zijn de kleinst mogelijke machtigingen die je kan instellen.
		Bieden de mogelijkheid om zeer nauwkeuring het toegangsniveau te
		bepalen.
	\item Het hoogste niveau, moleculaire of standaard machtigingen, omvat 6
		veel gebruikte combinaties van atomaire machtigingen.
\end{itemize}

\subsubsection{Instellen van NTFS machtigingen}

NTFS machtigingen toewijzen is niet moeilijk, maar in een grote omgeving met
veel bronnen en gebruikers moet je zeer methodisch en consistent te werk gaan.
De ACL editor, van een bestand of map, kan opgeroepen worden via de Security
tabpagina van het object. Hier kunnen ACEs toegevoegd en verwijderd worden. De
grijze slectievakjes duiden overgenomen machtigingen aan.

Deze inheritance of propagation is zeer handig om op een eenvoudige manier de
toegang wil definiëren tot de volledige naamruimte van een grote maphiërarchie.

De advanced knop in Security tabpagina geeft toegang tot een volgende niveau
bestaande uit 4 tabpagina's:
\begin{enumerate}
	\item Permissions: Toont steeds de correcte weergave v/d werkelijke
		ACLs en laat aanvullende mogelijkheden toe: Allow inheritance
		permissions, Replace all existing inheritance permissions,
		Add/Edit/Delete ACEs.
	\item Auditing: Laat toe om SACL van het object in te stellen.
	\item Owner: Laat toe om het ownership over te nemen. Het eigenaarschap
		kan niet eenzijdig worden overgedragen.
	\item Biedt de voor probleemdiagnose interessante mogelijk om voor een
		specifieke gebruiker het zelfde algoritme toe te passen dat de
		SRM gebruikt om de uiteindelijke machtigingen op een object na
		te gaan.
\end{enumerate}

\section{Wat gebeurt er met de machtigingen bij het verplaatsen van een bestand?
Wat gebeurt er met de machtigingen bij het kopiëren van een bestand?}

De gebruiker die de actie onderneemt wordt eigenaar van de bestanden wanneer ze
bij de bestemming aankomen.

Wanneer de bestemming een container is op een ander NTFS volume, of de bestanden
met standaard tools worden gekopieerd, vervallen de expliciete machtigingen. De
machtigingen van de doelcontainer worden overgenomen door het object zelf en al
zijn onderliggende objecten. Objecten die naar een niet NTFS volume worden
gekopiëerd, verliezen alle machtigingen.

Om de machtigingen tijdens een kopieropdracht te behouden moeten speciale tools
zoals robocopy en scopy gebruikt worden.

Wanneer bestanden of mappen worden verplaatst naar een container binnen
hetzelfde volume, worden de expliciete machtigingen behouden, en worden de
machtigingen van de container overgenomen.

\section{Op welke andere objecten zijn machtigingen van toepassing?}

Elk object in AD, elk object van een NTFS volume, elke registersleutel, elk
proces, en elke service heeft een security descriptor, en dus de bijhorende
machtigingen.

Het individueel instellen van de machtigingen van elk object zou heel wat werk
veroordzaken. Gelukkig worden de meeste objecten op één of andere manier
hiërarchisch gestructureerd en ken er gebruik gemaakt worden van overerving van
machtigingen.
