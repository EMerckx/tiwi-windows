\chapter{Gedeelde mappen en NTFS}

\section{Welke manier om gedeelde mappen te creeren biedt de meeste
configuratieinstellignen aan? Bespreek het doel van deze diverse instellingen en
de belangrijkste eigenschappen en mogelijkheden ervan.}

Om gedeelde mappen aan te kunnen maken, moet je over de juiste machtigingen
beschikken. Er zijn in Windows Server 2008+ diverse manieren om shares te maken:
\begin{itemize}
	\item In Explorer
	\item Shared folders snap-in
	\item Server Manager
	\item Command Prompt
\end{itemize}

De Server Manager biedt de meeste configuratie instellingen.

Je kan een nieuwe share maken door in Server Manager de Roles\textbackslash File
Services\textbackslash Share and Storage Management map te selecteren en
vervolgens de Provision Share taak uit te voeren. In de opeenvolgende dialogen
kunnen volgende zaken geconfigureerd worden:
\begin{itemize}
	\item Shared Folder Location is het pad tot de map die je wil delen
	\item NTFS Permissions
	\item SMB Permissions
	\item SMB Settings
	\item Quota Policy
	\item Filescreen Policy
	\item DFS Namespace publishing
\end{itemize}

\subsection{NTFS Permissions}

Na machtigingen op shares zijn NTFS machtigingen de tweede (beveiliging van
mappen) en derde (beveiliging van bestanden) verdedigingsmuren voor het
beveiligen van gegevens en netwerkbronnen. Er bestaan twee niveaus van NTFS
machtigingen:
\begin{itemize}
	\item Het laagste niveau bestaat uit 13 atomaire of speciale
		machtigingen, die de bouwstenen vormen voro het hogere niveau.
		Dit zijn de kleinst mogelijke machtigingen die je kan instellen.
		Bieden de mogelijkheid om zeer nauwkeuring het toegangsniveau te
		bepalen.
	\item Het hoogste niveau, moleculaire of standaard machtigingen, omvat 6
		veel gebruikte combinaties van atomaire machtigingen.
\end{itemize}

\subsection{SMB Permissions}

Share/SMB machtigingen vormen een eerste beveiligingslaag. Hebben overhand op
NTFS als ze meer beperkend zijn. Op share niveau kun je drie machtigingen
definiëren:
\begin{itemize}
	\item Full Control
	\item Read: bekijken en toepassingen uitvoeren
	\item Change: wijzigen, verwijderen. De eigenlijke share en NTFS
		machtigingen kunnen niet aangepast worden.
\end{itemize}

\subsection{SMB Setting}

\begin{description}
	\item[User limit] maximum aantal users die de share gezamelijk kunnen
		benutten
	\item[Acces based Enumeration] bestanden en submappen waar je geen
		enkele NTFS machtiging op hebt niet weergeven.
	\item[Client side caching] cachen van veel gebruikte netwerkbestanden.
\end{description}

\subsection{Quota policy}

Mogelijkheid om de hoeveelheid beschikbare ruimte voor gebruikers in te stellen.
\begin{description}
	\item[Volumequota] Gebaseerd op bestandseigendom en dus onmogelijk om
		per gebruikersgroep de som te laten controleren.
	\item[Mapquota] Vanaf Windows Server 2008 kan men wel mapquota op mappen
		plaatsen, maar dit geldt voor alle gebruikers. Werkt met soft en
		hard treshold.
\end{description}

\subsection{Filescreen Policy}

Mogelijkheid tot verhindern van opslaan van bestanden met een bepaalde extensie.
Er kunnen reacties getriggerd worden als iemand dit toch probeert.

\subsection{DFS Namespace publishing}

Distributed File System namespace is equivalent en fungeert zoals een share, met
het verschil dat een DFS naespace kan bestaan uit DFS mappen (die verwijst naar
een share, niet noodzakelijk op dezelfde server of zelf een DFS namespace),
bestanden en gewone mappen. Dit geeft de mogelijkheid om transparant aan
gebruikers bronnen van meerdere servers op 1 locatie aan te bieden. De DFS
topologie wordt automatisch in AD gepubliceerd en is dus altijd zichtbaar voor
gebruikers op alle servers in het domein.

DFS folder targets geven de mogelijkheid om identieke shares samen te nemen tot
1 DFS map. Een client zal dan bij het connecteren tot een DFS namespace elke
folder target uitproberen tot een werkende gevonden wordt. Dit geeft een hogere
fouttolerantie en een spreiding van de belasting. Synchronisatie van deze
verschillende shares gaat met FRS of DFS-R.

\section{Waar wordt de definitie en (partiële) configuratie van gedeelde mappen
opgeslagen? Hoe kan men deze wijzigen vanuit de command prompt?}

De configuratie wordt opgeslagen in het register van de server. Per share 1
multi-valued sleutel in de Shares subtak van de LanManServer service.

\begin{description}
	\item[net file] Bekijk en beheer bronnen die gedeeld worden over het
		netwerk. De server service moet draaien om dit commando te
		kunnen gebruiken.
	\item[net config] Bekijk het maximum aantal gebruikers die een gedeelde
		bron kunnen raadplegen en de maximum aantal open files per
		sessie.
	\item[net use] Verbind of verbreek de verbinding met een computer van
		een gedeelde bron.
	\item[net session] Beheer server connecties.
	\item[net share] Maak, verwijder, beheer en toon gedeelde bronnen.
	\item[net view] Toon informatie over de domeinen, computers en bronnen
		die gedeeld zijn door de aangegeven computer, inclusief de
		offline client cache setting.
	\item[net help] Bekijk de hulp pagina voor de netwerk commando's.

\end{description}

\section{Op welke diverse manieren kan men gebruik maken van gedeelde mappen?}

Eenvoudigste manier: rechtstreeks pad ingeven ind e adresbalk van een Explorer.
Als je een bepaalde netwerklocatie frequent gebruikt, is het aangewezen om deze
locatie in het lokale filesysteem te mappen. De share krijgt dan een
stationsletter toegewezen, en kan net als een lokaal station gebruikt worden.
Dit is in te stellen door in exploere rechtermuisknop op Network of My network
places -> Map Netwrok Drive -> wizard. Kan ook met het command net use.

\section{Geef een overzicht van de belangrijkste voordelen van de opeenvolgende
versies van het NTFS-bestandssysteem. Bespreek elk van deze aspecten (ondermeer
het doel, de voordelen en de beperkingen ervan), en geef aan hoe je er gebruik
kan van maken, bij voorkeur vanuit een Command Prompt.}

\subsection{Features vanaf v1.2}

\begin{itemize}
	\item Beveiliging op bestandsniveau
	\item Logging van schijfactiviteiten
	\item Dynamisch uitbreiden van partities/volumes
	\item Compressie
	\item Grotere partities, zonder performantiedegradatie
	\item Hardlinks
	\item Auditing op objecttoegang
\end{itemize}

\subsection{Features vanaf v3.0}

\begin{itemize}
	\item Reparsepunten en bestandssysteemfilters
	\item Transparante encryptie en decodering
	\item Individuele diskquota op volumeniveau
	\item Volumekoppelpunten: mounten van volumes in NTFS mappen
	\item Sparse bestanden
	\item Junction points naar mappen (soft link)
\end{itemize}
