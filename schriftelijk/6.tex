\chapter{Configuratie van gebruikersgroepen}

\section{Waar en hoe wordt het (volledige) lidmaatschap van een object tot een
groep bijgehouden? Op welke diverse manieren kan men dit lidmaatschap
configureren? Op welke diverse manieren kan men de volledige verzameling van
objecten, die er deel van uitmaken, achterhalen?}

In het properties venster van een gebruikersaccount kan men in het Member Of
tabblad lidmaatschap tot groepen configureren. Het memberOf attribuut dat
hiervoor gebruikt wordt, is gelinkt aan het member attribuut van de groep en is
hier de back-link. Het kan dus niet rechtstreeks gewijzigd worden, dit kan enkel
via het member attribuut van de groep.

In het properties venster van een groep kan men in het tabblad Members de groep
bevolken. In het Member Of tabblad kan men de groep zelf deel laten uitmaken van
een andere groep. De lijst van members wordt dus steeds opgeslagen in een
link-kenmerk met in de groep de forward-link members. En in het object dat deel
uitmaakt van de groep de back-link memberOf. Doordat memberOf een back-link
kenmerk is kan het niet rechtstreeks gewijzigd worden.

Groepsleden kunnen ook uit trusted domeinen geselecteerd worden. AD maakt
hiervoor een phantom object aan, dat het object uit het vertrouwde domein
vertegenwoordigt. Deze mapobjecten komen terecht in de container
ForeignSecurityPrincipals, en kunnen lid worden van lokale groepen in het
domein.

Uiteraard zijn de groepen ook te beheren via de Command Prompt door middel van
de opdrachten: net group, net localgroup, dsadd group, dsrm group, dsget group
en dsmod group.

De verzameling objecten die tot een groep behoren kan men verkrijgen met:
\begin{itemize}
	\item Door met het commando dsget group het attribuut members op te
		vragen van een groep.
	\item In het Members tabblad van de properties van de groep.
	\item Met behulp van een LDAP query.
\end{itemize}

\section{Door wie wordt het lidmaatschap van de diverse groepen bij voorkeur
ingesteld?}

Dit kan best door de beheerde gebeuren. Lidmaadschap bij een groep kan bepaalde
machtigingen met zich meebrengen en het is de taak van de beheerder om ervoor te
zorgen dat gebruikers niet meer rechten hebben dan dat ze nodig hebben.

In grote organisaties kan een deel van de beheerstaken gedelegeerd worden naar
andere personen. Zij zullen dan bv rechten hebben om een bepaalde subtak of OU
te beheren.

Bij subscriptiegroepen kunnen gebruikers zich zelf in en uitschrijven.

\section{Op welke diverse manieren kan men het beheer van van Active Directory
objecten, specifieke attributen van groepsobjecten in het bijzonder, delegeren
aan niet-administrators? Bespreek een aantal technieken om dit delegeren zo
eenvoudig mogelijk uit te voeren.}

Men kan in AD de beheerstaken van een OU delegeren aan een groep gebruikers.
Hiervoor moet men de ACLs van de OU instellen. Dit kan op drie manieren:
\begin{itemize}
	\item Via de Delegation of Control Wizard
	\item Via de properties van de OU
	\item Via de Command Prompt
\end{itemize}

\subsection{Delegation of Control Wizard}

Om de wizard op te starten klik je in dsa.msc met de rechtmuisknop op een OU en
selecteer je Delegate Control. Je selecteert de groepen of gebruikers waaraan je
beheersmachtigingen wil delegeren. Deze hoeven geen deel uit te maken van de
groep zelf. Het volgende dialoogvenster toont een lijst met de meest voorkomende
beheertaken die voor delegeren in aanmerking komen. Met create a custom task to
delegate kan je zelf een gedetailleerde keuze maken tussen alle beschikbare
delegeeropties. Het laatste dialoog venster toont ter bevestiging een overzicht
van de geselecteerde instellingen.

\subsection{OU properties}

Rechterklik op de OU, dan properties en naar het security tabblad gaan. Daar kan
je zowel taken delegeren als de instellingen voor reeds gedelegeerde taken
wijzigen. De lijst van molucaire machtigingen waaruit je kan kiezen is
afhankelijk van de objectklasse.

\subsection{Command Prompt}

Met de acldiag en dsacls opdrachten. Terug intrekken kan met dsrevoke.

\section{Aan welke groepen/entiteiten worden rechten in de praktijk toegekend?
Bespreek de bijzonderheden van dergelijke groepen/entiteiten, en vermeld er de
meest interessante voorbeelden van (telkens met hun bedoeling en hun
randeffecten).}

Rechten kunnen afzonderlijk aan een individuele gebruiker toegekend worden, maar
voor het organiseren van de beveiliging is het beter de gebruiker in een groep
te plaatsen, en te definieren welke rechten aan de groep toegekend worden.
Verschillende praktische groepen zijn oa:

\begin{description}
	\item[Backup Operators] hebben de bevoegdheid om bestanden en mappen te
		back-uppen en terug te plaatsen, zelfs als ze geen toestemming
		hebben om deze bestanden te lezen of te wijzigen.
	\item[Account Operator] maken en beheren gebruikersaccounts en groepen,
		en kunnen computers aan het domein toevoegen.
	\item[Server Operators] kunnen onder andere het systeem vanop afstand
		uitzetten, de systeemtijd veranderen, de harde schijf
		formatteren en mappen delen.
	\item[Print Operators] kunnen printers delen, wissen en beheren.
	\item[Administrators] hebben bijna elk recht. Toch krijgen beheerders
		met de ruime bevoegdheden van deze groep geen toegang tot alle
		bestanden en mappen.
	\item[User] kunnen programmas gebruiken, maar ze niet installeren.
\end{description}

AD kent ook impliciete groepen, deze hebben geen specifiek lidmaatschap en
kunnen op verschillende tijdstippen verschillende gebruikers vertegenwoordigen.
Enkele voorbeelden:
\begin{description}
	\item[Interactive] iedereen die de computer lokaal gebruikt.
	\item[Network] alle gebruikers die via het netwerk zijn verbonden met
		een computer.
	\item[Everyone] combinatie van de Interactive en Network groepen.
	\item[Authenticated Users] alle gebruikers die geauthentificeerd zijn.
	\item[System] het besturingssysteem, heeft bijna alle bevoegdheden.
		Processen van het OS hebben deze rechten nodig om te kunnen
		draaien.
\end{description}
