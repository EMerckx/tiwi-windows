\chapter{Active Directory server rollen}

Welke vragen moet men zich stellen na de initiële installatie van een Windows
Server toestel, in veband met bijzondere functies die de server kan vervullen
met betrekking tot Active Directory? Formuleer bij het beantwoorden van deze
vragen telkens (voor zover relevant):
\begin{itemize}
	\item Hoe bepaald wordt welke servers een dergelijke specifieke functie
		vervullen?  Hoeveel zijn er nodig (in termen van:
		minimaal/exact/maximaal aantal, in functie van ...), en waarom?
	\item Eigenschappen zoals bedoeling, noodzaak, kriticiteit, inhoud,
		synchronisatie, voor welke Windows versie(s) van toepassing,...?
	\item De eventuele relatie tussen de diverse functies. Vermeld
		bijvoorbeeld welke functies al dan niet door dezelfde server
		kunnen vervuld worden, of misschien juist wel door dezelfde
		server moeten vervuld worden.
	\item Op welke diverse manieren men de toewijzing van elke bijzondere
		functie kan instellen, wijzigen en/of ongedaan maken?
\end{itemize}

\section{Wordt server al dan niet opgenomen in een domein?}

Als een Windows Server toestel lid is van een werkgroep, niet van een domein,
dan wordt het een zelfstandige server genoemd. Deze profiteren niet van de
voordelen die AD biedt. Vooral indien er reeds een AD domeinstructuur aanwezig
is, is de keuze snel gemaakt.

\section{Vervult de in een domein opgenomen server al dan niet de functie van
domeincontroller?}

Een in een domein opgenomen server die de functie van domeincontroller niet
vervult, wordt een lidserver (member server) genoemd. Meestal fungeren
lidservers als file servers, toepassingsservers, database servers, web servers,
firewalls, routers of een combinatie hiervan. In Windows Server 2008+ worden
dergelijke functies gegroepeerd op drie niveau's:
\begin{description}
	\item[Server rollen] (DNS, DHCP, ...) en Web Server implementeren
		primaire serverfuncties.
	\item[Role services] optionele, meer complexe, server rollern.
	\item[Features] (Powershell, SNMP, backup, ...) zorgen voor meer
		ondersteunende functies.
\end{description}
Men kan aanvullende server rollen, rol services en features configureren via
respectievelijk de Add Roles, Add Role Services en Add Features wizards van
Server Manager. Dit kan ook met de ServerManagerCmd opdracht.

Zowel voor lidservers als voor domeincontrollers gelden groepsbeleidinstellingen
die gedefinieerd zijn voor het domein, de OUs of de site. Lidserver behouden, in
tegestelling tot domeincontrollers, een eigen lokale beveiligingsdatabank, de
Security Account Manager (SAM).
Doorgaans configureert men slechts een fractie van de Windows Server toestellen
als domeincontroller. Minimaal 1 per domein.

\section{Indien gekozen wordt voor domeincontroller, moet ook de functie van
globale catalogus ondersteund worden?}

Indien er slechts een domein in het forest is, mogen alle domeincontrollers tot
global catalog server worden gepromoveerd. Om het replicatieverkeer te beperken,
zorgt me er best voor dat er steeds een global catalog server per site aanwezig
is. De promotie gebeurt door de global catalog optie aan tevinken in de general
tabpagina van de properties van de NSDS settings van een domeincontroller. Dit
is terug te vinden in de AD sites and services snap-in. De global catalog server
heeft een kopie van alle objecten van de domeingegevens van het domein waarin de
global catalog server zich bevindt, en een kopie van een subset van de
eigenschappen van alle objecten van het gehele forest. Deze subset wordt enkel
gerepliceerd tussen GC servers. Verder bevat de GC nog een kopie van de
configuratiegegevens van alle domeinen in het forest, en het unieke schema van
het forest. Eventueel worden ook nog specifieke applicatiepartities bijgehouden
in de global catalog.

\section{Welke domain controllers worden als RODC ingesteld?}

Domeincontrollers configureren als Read Only Domain Controller is goede praktijk
als de fysieke beveiliging van het toestel niet kan worden gegarandeerd, of waar
specifieke interactieve toepassingen enkel op een DC kunnen worden uitgevoerd.
Een bijkomend voordeel is dat het replicatieverkeer in één enkele richting
wordt beperkt. Er kan ook nog configuratie van een filtered attribute set (welke
kenmerken worden gerepliceerd, en welke niet) en een password replication policy
worden geconfigureerd met credential caching voor specifieke gebruikers en
computers. RODC's kunnen tevens de rol van GC en DNS server vervullen. In het
geval van DNS gaat het over een ordinaire secundaire nameserver. Een RODC biedt
ook ondersteuning voor de operations masters rollen. Er geldt wel de beperking
dat een RODC enkel met Windows Server 2008 DC's gegevens kan repliceren.

\section{Welke domeincontrollers vervullen de operations master rollen?}

Een aantal specifieke AD functies kunnen slechts door één enkele controller in
het domein of in het forest vervuld worden.

\subsection{Rollen uniek in elk domein}
\begin{description}
	\item[RID master] In domeinen met minimaal Windows 2000 native
		functioneel niveau wijst de RID master reeksen relatieve IDs
		toe aan alle domeincontrollers in het domein. Hiermee kunnen de
		domeincontrollers zelfstandig SPN objecten met een SID
		aanmaaken. Wanneer een domeincontroller 80\% van zijn RID pool
		heeft opgebruikt, vraagt hij een nieuwe reeks aan bij de RID
		master.
		Een object tussen domeinen verplaatsen moet gebeuren vanop de
		RID master van het domein dat het object bevat.
	\item[PDC emulator master] In een Windows 2000 mixed domein met NT4
		back-up domeincontrollers fungeert de PDC emulator master als
		een volledige emulatie van een NT4 primaire domeincontroller.
		NT4 domeinen kunnen hierdoor services blijven verlenen zonder te
		weten dat hun PDC in feite door AD vervangen is. De PDC emulator
		master verwerkt wachtwoordwijzigingen van cliënts en repliceert
		bijgewerkte gegevens naar de NT4 back-up domeincontrollers. Hij
		krijgt ook een voorkeursbehandeling bij de replicatie van
		wachtwoordwijzigingen, uitgevoerd door andere domeincontrollers
		in het domein.
		De PDC emulator master fungeert ook als primaire bron van
		tijdssynchronisatie. Best RID master en PDC emulator master
		laten vervullen door dezelfde domeincontroller. Het verlies van
		de PDC emulator heeft gevolgen voor netwerkgebruikers.
	\item[infrastructure master] is verantwoordelijk voor het bijwerken van
		verwijzingen vanuit objecten in het eigen domein naar objecten
		in andere domeinen. De infrastructure master van een domein
		vergelijkt continu de kenmerken van zijn phantom objecten met de
		kenmerken van de corresponderende objecten in externe domeinen,
		en de kenmerken van phantom objecten in externe domeinen die
		dooverwijzen naar eigen objecten met de kenmerken van deze
		objecten. Hij doet hiervoor een beroep op de global catalog.
		Omdat domeincontrollers die global catalog zijn geen phantom
		objecten aanmaken, doordat ze reeds beschikken over een kopie
		van de objecten van andere domeinen, moet de rol van
		infrastructure master vervuld worden door een domeincontroller
		die geen global catalog is.
\end{description}

\subsection{Rollen uniek in het forest}
\begin{description}
	\item[schema master] beheert alle gewijzigde gegevens voor het schema.
		Als je het schema van een forest wil bijwerken, dan moet je dit
		doen via de schema master. Dit om conflicterende wijzigingen in
		het schema te vermijden.
	\item[domain naming master] beheert het toevoegen en verwijderen van
		domeinen en applicatiepartities in het forest. Het is de enige
		domeincontroller die de partitions container van de configuratie
		gegevens kan wijzigen. Het moet een global catalog controller
		zijn.
\end{description}
