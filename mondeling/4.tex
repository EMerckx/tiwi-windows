\chapter{Active DIrectory functionele niveaus}

\section{Geef de diverse functionele niveaus waarop Active Directory kan
ingesteld worden, en welke beperkingen er het gevolg van zijn.}

Elk domein is gekenmerkt door een bepaald domein functioneel niveau. Dit domein
functioneel niveau geeft aan welke minimum eis er gesteld wordt aan het
besturingssysteem van de domeincontrollers, en bepaalt tegelijkertijd welke
faciliteiten er beschikbaar zijn. Het wordt opgeslagen in twee kenmerken van het
domeinobject: ntMixedDomain en msDS-Behavior-Version.
\begin{description}
	\item[Windows 2000 mixed domein functioneel niveau] kan zowel NT4,
		Windows 2000 Server als Windows server 2003 domeincontrollers
		bevatten, maar biedt de laagste AD functionaliteit. Het kan geen
		Windows Server 2008+ domeincontrollers bevatten. NT5 domeinen
		worden standaard op dit niveau gezet, omdat ze ervan uitgaan dat
		er nog NT4 domeincontrollers aanwezig kunnen zijn.
	\item[Windows 2000 native domein functioneel niveau] biedt de keuze
		tussen willekeurige NT 5+ domeincontrollers. Stelt geen eisen
		aan lidservers en werkposten.
	\item[Windows Server 2003 domein functioneel niveau] er zijn enkel
		Windows Server 2003+ domeincontrollers mogelijk. Stelt geen
		eisen aan lidservers en werkposten.
	\item[Windows Server 2008 domein functioneel niveau] er zijn enkel
		Windows Server 2008+ domeincontrollers mogelijk. Stelt geen
		eisen aan lidservers en werkposten.
\end{description}

Analoog aan het domain functioneel niveau, is er ook een forest functioneel
niveau. Dit wordt opgeslagen in het msDS-Behavior-Version kenmerk van het
partitions containerobject van de configuratie gegevens.
\begin{description}
	\item[Windows 2000 forest functioneel niveau] stelt geen enkele eis aan
		het functioneel niveau van de liddomeinen en is de standaard
		instelling.
	\item[Windows 2003 forest functioneel niveau] kan enkel domeinen
		bevatten die minimaal op Windows Server 2003 domein functioneel
		niveau staan.
	\item[Windows 2008 forest functioneel niveau] kan enkel domeinen
		bevatten met Windows Server 2008 domein functioneel niveau.
\end{description}

\section{Bespreek van elk niveau alle eraan gekoppelde voordelen. Geef hierbij
telkens een korte bespreking van de ingevoerde begrippen.}

\subsection{Domein functioneel niveau}

Windows 2000 mixed domein biedt het minst AD functionaliteit. We bespreken voor
de opeenvolgende domein functioneel niveaus steeds welke functionaliteit er is
bijgekomen tegen over het vorige niveau.

\subsubsection{Windows 2000 native}

De voordelen van dit niveau tov Windows 2000 mixed functioneel niveau:
\begin{itemize}
	\item Eén enkele global catalog voor het ganse forest volstaat. De
		Global Catalog bevat een read-only en verkorte inhoudopgave van
		elk domein in het forest.
	\item Transitieve vertrouwensrelaties tussen verschillende domeinen van
		eenzelfde forest. Tussen twee domeinen kan een
		vertrouwensrelatie tot stand gebracht worden, zodat gebruikers
		in het ene, trusted (vertrouwd), domein kunnen geverifieerd
		worden door een domeincontroller in het andere, trusting
		(vertrouwend), domein.
	\item Alle domeincontrollers kunnen zelfstandig een aantal SPN objecten
		aanmaken. Aan SPN objecten moet een uniek SID toegekend worden.
		De RID master wijst hiervoor aan elke domeincontroller reeksen
		relatieve IDs toe. Hievoor (Windows 2000 mixed) moest men steeds
		een beroep doen op de PDC emulator master.
	\item Ruimere mogelijkheden om gebruikers en/of computers te verzamelen
		in groepen, met bovendien minder beperkingen op de conversie, de
		zichtbaarheid en het nesten van deze groepen. Een
		veiligheidsgroep kan bv omgevormt worden in een distributiegroep
		en vice versa. Veiligheidsgroepen zijn groepen waaraan je
		rechten en machtigingen kan toekennen. Distributiegroepen hebben
		geen enkele beveiligingsfunctie. Ook universele groepen (kunnen
		leden hebben uit elk domein van het forest) zijn nu mogelijk.
	\item Alle SIDs die een SPN object in het verleden gehad heeft, worden
		bijgehouden in het sIDHistory kenmerk.
\end{itemize}

\subsubsection{Niveau 2: Windows Server 2003}

De voordelen van dit niveau tov Windows Server 2000 Native fucntioneel niveau:
\begin{itemize}
	\item Aanvullende schema klassen en attributen. Zoals bv
		lastLogonTimestamp, een over de domeincontrollers
		gesynchroniseerde versie van het tijdstip van laatste logon van
		UPN objecten. Schema klassen en attributen zijn een formele
		beschrijving van de objecten in AD en hun kenmerken.
	\item Laat toe om een domeincontroller van naam te veranderen zonder
		degradatie en promotie.
	\item Aanvullende opdrachten. Zoals bv redirusr en redircmp om de
		default AD container te wijzigen waarin respecievelijk nieuwe
		gebruikers en nieuwe computers terecht komen.
	\item Caching op domeincontroller niveau van UPN suffices en het
		lidmaatschap van universele groepen, sodat het niet meer strikt
		noodzakelijk is dat tijdens het inlogproces een global catalog
		bereikbaar is.
	\item Group policies filteren, op basis van beveiligingsgroepen, maar
		ook met behulp van WMI scripts. Filteren is verhinderen dat het
		GPO op specifieke groepen van gebruikers en computers toegepast
		wordt.
\end{itemize}

\subsubsection{Niveau 3: Windows Server 2008}

De voordelen van dit niveau tov Windows Server 2003 functioneel niveau:
\begin{itemize}
	\item Aanvullende schema klassen en objecten. Waardoor ondermeer de
		werkpost, waarop een gebruik het laatst zich met succes aagemeld
		heeft, en het aantal mislukte pogingen sindsdien, opgevraagd
		kan worden.
	\item Encryptie van het Kerberos protocol met langere sleutels.
	\item Fijnkorrelig wachtwoordbeleid (fine-grained password policies),
		zodat wachtwoordenrestricties niet langer globaal zijn voor het
		gehele domein, maar specifiek ingesteld kunnen worden voor
		individuele gebruikers of gebruikersgroepen.
	\item Replicatie van DFS namespaces en van de SYSVOL share met behulp
		van DFS Replication, wat meer performanter is dan via de
		traditionele File Replication Service. Distributed File System
		(DFS) biedt een mechanisme voor het omleiden van shares, door ze
		te bundelen in één naamruimte. Als dezelfde share beschikbaar
		gesteld wordt op verschillende servers, moet er replicatie
		hiertussen voorzien worden. DFS-R is ontworpen met de
		minimalisering van de netwerkbeslasting als voornaamste doel en
		is dus efficienter op vlak van netwerkverkeer.
		De SYSVOL share bevat de logon scripts (configuratie bij
		inloggen) en wordt mbv DFS-R gerepliceerd naar alle
		domeincontrollers.
\end{itemize}

\subsection{Forest functioneel niveau}

Windows 2000 forest functioneel niveau biedt het minst AD functionaliteit en is
de standaar instelling voor nieuwe installaties. We bespreken voor de
opeenvolgende forest functioneel niveaus steeds welke functionaliteit er is
bijgekomen tegen over het vorige niveau.

\subsubsection{Niveau 2: Windows 2003}

De voordelen van tov Windows 2000 forest functioneel niveau:
\begin{itemize}
	\item Het hergebruiken van gedeactiveerde attributen en klassen.
	\item Dynamische hulpklassen. Bij creatie van een object kan er gekozen
		worden welke hulpklasse deze gebruikt.
	\item Dynamische objecten, met een beperkte levensduur, die na het
		verstrijken van de entryTTL waarde automatisch uit AD verwijderd
		worden, tenzij ze opgefrist worden.
	\item Efficiënte replicatie van de global catalog gegevens, waardoor
		ondermeer toevoeging van een nieuw kenmerk niet langer leidt tot
		een volledige synchronisatie van alle objectkenmerken.
	\item Veranderen van de naamgeving en de hiërarchische structuur van
		domeinen in een forest.
	\item Transitieve vertrouwensrelaties tussen verschillende forests.
	\item Read-only Windows Server 2008+ domeincontrollers. Ideaal voor
		locaties waar de domeincontroller fysiche niet beveiligd kan
		worden. Er kunnen geen aanpassingen van de AD gegevens gebeuren
		op deze domeincontroller.
	\item Efficiëntere KCC algoritmen voor het construeren van de
		replicatietopologie.
	\item Replicatie van de individuele waarden van multi-valued attributen.
\end{itemize}

\subsubsection{Niveau 3: Windows 2008}

Dit niveau biedt geen aanvullende functionaliteit tov Windows Server 2003 forest
functioneel niveau. Toch is het om beveiligingsredenen van belang om op dit
niveau over te schakelen, bijvoorbeeld indien met Read Only Domain Controllers
(RODCs) met RODC filtered attribtue set introduceert. Dit laatste is een
verzameling van kenmerken die niet naar een RODC gerepliceerd worden.

\section{Hoe kan men detecteren op welk niveau een Active Directory omgeving
zicht bevindt?}

Het domein functioneel niveau wordt opgeslagen in twee kenmerken van het
domeinobject: \begrip{ntMixedDomain} en \begrip{msDS-Behavior-Version}. Het
ntMixedDomain kenmerk zal enkel bij Windows 2000 mixed domeinen op 1 staan, bij
alle andere niveau's op 0. Het msDS-Behavior-Version kenmerk voor Windows 2000
mixed en Windows 2000 native op 0, voor Windows Server 2003 op 2 en voor Windows
Server 2008 op 3. Dit kan bv met dsquery opgevraagd worden.

Het forest functioneel niveau wordt opgeslagen in een kenmerk van het partitions
containerobject van de configuratiegegevens: \begrip{msDS-Behavior-Version}.
Voor Windows 2000 forest functioneel niveau staat dit op 0, voor Windows Server
2003 forest functioneel niveau op 2 en voor Windows Server 2008 forest
functioneel niveau op 3.

\section{Op welk diverse manieren kan men het functionele niveau verhogen of
verlagen?}

De omschakeling naar een bepaald domein of ferst functioneel niveau moet manueel
gebeuren, gebeurt niet automatisch van zodra de controllers of domeinen aan een
minimum voorwaarde voldoen.
\begin{itemize}
	\item Door zelf manueel de attributen te manipuleren
	\item Via de Active Directory Domains and Trust snap-in
\end{itemize}
Men kan enkel het functioneel niveau verhogen, niet verlagen. De wijzigingen
worden pas effectief doorgevoerd na een heropstart van alle domeincontrollers.
Achteraf geinstalleerde domeincontrollers zullen automatisch functioneren op het
huidige niveau.
