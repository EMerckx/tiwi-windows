\chapter{Active DIrectory functionele niveaus}

\section{Geef de diverse functionele niveaus waarop Active Directory kan
ingesteld worden, en welke beperkingen er het gevolg van zijn.}

Elk domein is gekenmerkt door een bepaald domein functioneel niveau. Dit domein
functioneel niveau geeft aan welke minimum eis er gesteld wordt aan het
besturingssysteem van de domeincontrollers, en bepaalt tegelijkertijd welke
faciliteiten er beschikbaar zijn. Het wordt opgeslagen in twee kenmerken van het
domeinobject: ntMixedDomain en msDS-Behavior-Version.
\begin{description}
	\item[Windows 2000 mixed] kan zowel NT4, Windows 2000 Server als Windows
		server 2003 domeincontrollers bevatten, maar biedt de laagste AD
		functionaliteit. NT5 domeinen worden standaard op dit niveau
		gezet, omdat ze ervan uitgaan dat er nog NT4 domeincontrollers
		aanwezig kunnen zijn.
	\item[Windows 2000 native] biedt de keuze tussen willekeurige NT 5+
		domeincontrollers.
	\item[Windows Server 2003] er zijn enkel Windows Server 2003+
		domeincontrollers mogelijk.
	\item[Windows Server 2008] er zijn enkel Windows Server 2008+
		domeincontrollers mogelijk.
\end{description}

Analoog aan het domain functioneel niveau, is er ook een forest functioneel
niveau. Dit wordt opgeslagen in het msDS-Behavior-Version kenmerk van het
partitions containerobject van de configuratie gegevens.
\begin{description}
	\item[Windows 2000] stelt geen enkele eis aan het functioneel niveau van
		de liddomeinen en is de standaard instelling.
	\item[Windows 2003] kan enkel domeinen bevatten die minimaal op Windows
		Server 2003 domein functioneel niveau staan.
	\item[Windows 2008] kan enkel domeinen bevatten met Windows Server 2008
		domein functioneel niveau.
\end{description}

\section{Bespreek van elk niveau alle eraan gekoppelde voordelen. Geef hierbij
telkens een korte bespreking van de ingevoerde begrippen.}

Domein functioneel niveau
\begin{description}
	\item[Windows 2000 native] 1 global catalog voor heel forest,
		Automatisch wederzijdse vertrouwensrelaties tussen alle domeinen
		van een forest, Alle domeincontrollers kunnen zelfstandig een
		aantal SPN-objecten aanmaken
	\item[Niveau 2: Windows Server 2003] Aanvullende schema klassen en
		attributen
	\item[Niveau 3: Windows Server 2008] Kerberos encryptie met langere
		sleutels, Fine-grained password policies
\end{description}

Forest functioneel niveau
\begin{description}
	\item[Niveau 0: Windows 2000] Stelt geen enkele eis aan het domein
		functioneel niveau van het liddomein
	\item[Niveau 2: Windows 2003] Hergebruik van gedeactiveerde
		schemaobjecten, Dynamisch gebruik van hulpklassen
	\item[Niveau 3: Windows 2008] Geen site coverign probleem bij sites met
		enkel RODC's
	\item[Niveau 4: Windows 2008 R2] Online restore mogelijkheid van
		thombstone objects
\end{description}

\section{Hoe kan men detecteren op welk niveau een Active Directory omgeving
zicht bevindt?}

Het domein functioneel niveau wordt opgeslagen in twee kenmerken van het
domeinobject: \begrip{ntMixedDomain} en \begrip{msDS-Behavior-Version}. Het
ntMixedDomain kenmerk zal enkel bij Windows 2000 mixed domeinen op 1 staan, bij
alle andere niveau's op 0. Het msDS-Behavior-Version kenmerk voor Windows 2000
mixed en Windows 2000 native op 0, voor Windows Server 2003 op 2 en voor Windows
Server 2008 op 3. Dit kan bv met dsquery opgevraagd worden.

Het forest functioneel niveau wordt opgeslagen in een kenmerk van het partitions
containerobject van de configuratiegegevens: \begrip{msDS-Behavior-Version}.
Voor Windows 2000 forest functioneel niveau staat dit op 0, voor Windows Server
2003 forest functioneel niveau op 2 en voor Windows Server 2008 forest
functioneel niveau op 3.

\section{Op welk diverse manieren kan men het functionele niveau verhogen of
verlagen?}

De omschakeling naar een bepaald domein of ferst functioneel niveau moet manueel
gebeuren, gebeurt niet automatisch van zodra de controllers of domeinen aan een
minimum voorwaarde voldoen.
\begin{itemize}
	\item Door zelf manueel de attributen te manipuleren
	\item Via de Active Directory Domains and Trust snap-in
\end{itemize}
Men kan enkel het functioneel niveau verhogen, niet verlagen. De wijzigingen
worden pas effectief doorgevoerd na een heropstart van alle domeincontrollers.
Achteraf geinstalleerde domeincontrollers zullen automatisch functioneren op het
huidige niveau.
