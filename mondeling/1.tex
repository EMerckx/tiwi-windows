\chapter{Structuur van Active Directory gegevens}

\section{Bespreek de diverse namen die alle Active Directory objecten
identificeren.}

\begin{description}
	\item[Relative Distinguished Name (RDN)] zorgt voor de unieke
		identificatie binnen de container waar het object zich in
		bevindt. bv \begrip{cn=beelzebub} is RDN van computer beelzebub.
		De RDN wordt opgeslagen in het kenmerk \begrip{cn} van het
		object.
	\item[Distinguished Name (DN)] is opgebouwd uit de RDN van het object
		zelf en de RDNs van alle containerobjecten waarvan het object in
		de hiërarchische indeling deel van uitmaakt. Dit zorgt voor een
		unieke naamgeving voor elk object. Elk opeenvolgend deel van de
		DN heeft de vorm \begrip{attribuut=waarde}, staat bekend als
		\engels{attributed naming}. De delen worden gescheiden door een
		komma. De DN naamgeving werkt van beneden (objectnaam) naar
		boven (root van AD). bv
		\begrip{cn=beelzebub,ou=iii,dc=hogent,dc=be}
		De DN is belangrijk voor de werking van het LDAP protocol, bv
		belangrijk bij het programmeren van scriptcode die AD
		rechtstreeks aanspreekt. Een LDAP-cliënt kan query's op
		willekeurige objecten uitvoeren door een LDAP URL te gebruiken
		met de vorm \begrip{LDAP://server DNS naam/object DN}. De DN
		wordt opgeslagen in het kenmerk \begrip{distinguishedName} van
		het object.
	\item[Canonical name] wordt op dezelfde manier samengesteld als de DN,
		maar wordt op een eenvoudigere manier weergegeven:
		\begrip{hogent.be/iii/beelsebub}. De meeste hulpmiddelen in
		Active Directory tonen de canonieke naam, in plaats van de DN.
		Het wordt opgeslagen in het kenmerk \begrip{canonicalName} van
		het object.
	\item[Globaal Uniek ID (GUID)] is een 128-bit getal dat niet kan
		gewijzigd worden. Het wordt bepaald bij de creatie van het
		object. Ook als het object verplaatst of gewijzigd wordt zal de
		GUID niet veranderen. De GUID is beschikbaar voor verwijzing van
		externe processen en programmeerfuncties. Het wordt opgeslagen
		in het kenmerk \begrip{objectGUID} van het object.
\end{description}

\section{Wat zijn SPN objecten? Bespreek de aanvullende naamgeving voor deze
objecten.}

Security Principal Objects (SPN) zijn AD-objecten waaraan Security ID's (SID)
zijn goegewezen. De SID wordt opgeslagen in het \begrip{objectSid} kenmerk.
SPN objecten worden, na aanmelding op het netwerk, gebruikt voor het verlenen
van toegang tot domeinbronnen. Ze zijn van toepassing op gebruikeraccounts,
computeraccounts, groepen en domeinen. SIDs zijn uniek, ook in de tijd: nieuw
aangemaakte accounts kunnen zo nooit de rechten krijgen van een oud account.

SIDs worden meestal voorgesteld dooe een hiërarchische string getallen
gescheiden door koppeltekens. bv \begrip{S-1-5-x-y-z-500} S-1-5 is de standaard
prefix, bestaande uit een \engels{Revision Level} en een \engels{Authority
Identifier}. X, y en z zijn 32-bit getallen specifiek voor het domein
(\engels{Domain Subauthority Identifier}). 500 is een \engels{relative ID}
(RID), dat nar het feitelijk object verwijst.

Bij verplaatsing binnen hetzelfde domein blijft de SID ongewijzigd. Bij
verplaatsing naar een ander domein zal de SID we veranderen. De GUID blijft
altijd ongewijzigd. Om te vermijden dat een gebruiker hierna zijn toegang tot
domeinbronnen zou verliezen worden alle vorige SIDs van het object bijgehouden
in het \begrip{sIDHistory} kenmerk.

\subsection{Aanvullende naamgeving voor gebruikersaccounts}

De RDN, DN en canonieke naam zijn ongeschikt als aanmeldingsnaam omdat ze bij
verplaatsing veranderen. SID en GUID zijn mits hun numerieke karakter ook niet
geschikt hiervoor. Elke gebruikersaccount heeft ook een \engels{User Principal
Name} (UPN), aka aanmeldingsnaam. Wordt bij het aanmaken van het account
ingevoerd door de beheerder en opgeslagen in het \engels{userPrincipalName}
kenmer. Dit moet uniek zijn binnen het forest. UPN bestaat standaard uit de RDN
van de gebruiker en de UPN suffix, gekoppeld met het @-teken.
Voor de UPN suffix zijn er een aantal alternatieven:
\begin{itemize}
	\item Standaard wordt de DNS naam van het domein genomen
	\item Dikwijlst kiest men voor de DNS naar van het root domein van het
		forest
	\item Kan ook een willekeurige alternatieve naam zijn, als deze op
		voorhand geregistreerd is door een beheerder.
\end{itemize}

UPN is de meest geschikte aanmeldingsnaam maar wordt maar zelden gebruikt door
compatibiliteit met vroegere NT versies. Meestal wordt als aanmeldingsnaam de
NetBIOS naam van het domein en de SAM accountnaam van de gebruiker, gekoppeld
met \, genomen. De NetBIOS naam bestaat uit max 15 karakters, standaard de meest
linkse component van de DNS naam. De SAM accountnaam bestaat uit max 20
karakters, standaard de eerste 20 bytes van de RDN. NetBIOS naam moet uniek zijn
in het forest, SAM accountnaam uniek in het domein en wordt opgeslagen in het
\begrip{sAMAccountName} kenmerk

\subsection{Aanvullende naamgeving voor computeraccounts}

Elke computeraccount in AD heeft behalve zijn RDN, DN, canonieke naam, GUID en
SID ook een SAM accountnaam (ook NetBIOS computernaam genoemd), een DNS hostnaam
en een \engels{Service Principal Name} (SPN).
SAM naam bestaat uit eerste 15 bytes van RDN, gevolgd door een \$, en wordt
opgeslagen in het sAMAccountName kenmerk.
De DNS hostname wordt opgeslagen in het dnsHostName kenmerk en bestaat
standaard uit de eerste 15 tekens van de RDN en de suffix van de primaire DNS.
De SPN is essentieel tijdens de wederzijdse verificatie van client software en
de server die een bepaalde service biedt: de client zoekt een computeraccount op
aan de hand van de SPN naam van de service waarmee hij een verbinding tot stand
wil brengen. De SPN wordt bepaald door het multi-valued
\engels{servicePrincipalName} kenmerk, oa samengesteld op basis van DNS naame en
eventueel SRV records die naar het toestel verwijzen.

\section{Enkele veel gebruikte klassen vertonen nog meer identificerende
attributen voor hun instanties. Bespreek deze klassen en attributen}

\begin{itemize}
	\item Common name (cn)
	\item IDAPDisplayName
	\item Object identifier (OID)
\end{itemize}

\section{In welke partities is de Active Directory informatie verdeeld? Geef de
betekenis van elke partitie, hun onderlinge relatie, en de
replicatiekarakteristieken erven.}

De directory wordt enkel opgeslagen op domeincontrollers. Op elke
domeincontroller wordt een kopie opgeslagen van de directory voor het domein
waarin de controller zich bevindt. De informatie die in de directory is
opgeslagen, is fysiek verdeeld in minimaal drie categorieën: \begrip{directory
partities}
\begin{description}
	\item[Domaingegevens] bevatten de eigenlijke informatie over objecten in
		het domein (zoals gedeelde bronnen, gebruiker- en
		computeraccounts). Bij de installatie worden een aantal
		standaard objecten geinstalleerd door het systeem, bv de
		beheerderaccount. Wordt gedeeld per domein, dus er zijn evenveel
		partities met domeingegevens als er domeinen zijn in het forest.
		De domeingegevens worden niet gedistribueerd naar andere
		domeinen. Er wordt we een subset van alle kenmerken van alle
		domeinen bijgehouden in de globale catalogus.
	\item[Configuratiegegevens] beschrijven de fysieke topologie van de
		directory. Dit bevat onder andere een lisjt van alle
		domeinstructuren, de locaties van de domeincontrollers en de
		global catalog controllers, de sites en de replicatietopologie.
		De meeste gemeenschappelijke instellingen van het ganse forest
		worden hierin opgeslagen, als kenmerken van objecten. De
		configuratiegegevens gelden voor alle domeinen in het forest.
	\item[Het schema] is de formele definitie van alle objecten en
		kenmerkgegevens die kunnen opgeslagen worden in de directory.
		Dit schema is uniek voor alle domeinen in het forest.
	\item[Applicatiepartities] dit is een 4e vorm van partities die vanaf
		Windows server 2003 kan aangemaakt worden. Een deel van de AD
		gegevens kan in één of meerdere gescheiden applicatiepartities
		worden ondergebracht, dit is interessant voor dynamische
		objecten. Ze kunnen geen SPN objecten bevatten.  Objecten binnen
		de applicatie partitie kunnen niet verplaatst worden buiten de
		partitie. bv voor DNS is er een aparte applicatiepartitie.
\end{description}

Het schema en de configuratiegegevens zijn containerobjecten in het root domein
van het forest. Ook de eventuele applicatiepartities vormen containerobjecten
ten opzichte van de overeenkomstige domeingegevens. Dit is een logische
structurering, fysische bestaat het wel uit meerdere bestanden.

AD servers wisselen continu gegevens uit. Elke partitie in de directory vormt
een aparte eenheid voor replicatie, waarbij telkens een specifieke groep
controllers hoort. Het schema en configuratiegegevens worden gerepliceerd naar
alle domeincontrollers in het forest. De domeingegevens worden logischerwijs
binnen het domein gerepliceerd. Applicatiepartities worden gerepliceerd tussen
een eigen deelverzameling specifiek hiervoor geconfigureerde domeincontrollers
van het forest. DNS partities kunnen zo bv enkel naar DNS nameservers
gerepliceerd worden.

De koppeling tussen een partitie en zijn replicerende domeincontrollers wordt
bijgehouden in het \begrip{msDS-NC-Replica-Locations} kenmerk van het
overeenkomstige crossref object in de configuratiegegevens.

Een subset van de kenmerken van alle objecten in de domeingegevens van elk
domein in het forest worden gerepliceerd naar de globale catalogus.
