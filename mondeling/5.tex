\chapter{Active Directory domeinstructuren}

\section{Wat is de bedoeling van vertrouwensrelaties?}

Tussen 2 domeinen kan een vertrouwensrelatie tot stand gebracht worden, zodat de
gebruikers in het ene, trusted domein kunnen geverifieerd worden door de
domeincontroller in het andere, trusting domein. Vertrouwensrelaties worden
weergegeven met een pijl in de richting van het trusted domein. Een gebruiker
kan pas toegang krijgen tot bronnen in een ander domein als er een
vertrouwenspad is van het trusting domein naar het trusting domein. Een
vertrouwenspad is een continue rij vertrouwensrelaties tussen domeinen. Dat een
gebruiker door een trusting domeincontroller is geverifieerd, wil nog niet
automatisch zeggen dat de ggebruiker toegang heeft tot de bronnen in dat domein.
Deze toegang wordt geregeld met gebruikersrechten en machtigingen die aan de
gebruiker toegekend zijn door de domeinbeheerder van het trusting domein.

\section{Bespreek de verschillende soorten vertrouwensrelaties.}

\subsection{Automatische vertrouwensrelaties}

Windows Server maakt automatisch vertrouwensrelaties aan tussen domeinen en hun
child domeinen. Deze vertrouwensrelaties kunnen niet verbroken worden en zijn
automatisch bi-directioneel en transitief. Windows Server maakt ook automatisch
vertrouwensrelaties aan tussen de trees van eenzelfde forest: de root domeinen
van alle trees in het forest vormen transitieve vertrouwensrelaties met het
forest root domein van het forest.
Aangezien Windows Server vertrouwensrelaties bi-directioneel en transitief zijn,
heeft een domein dat nieuw aangemaakt wordt in een tree of forest, automatisch
vertrouwensrelaties met alle andere Windows Server domeinen in de tree of het
forest.
Om in NT 4 hetzelfde te bekomen, moet je als systeembeheerder zelf
vertrouwensrelaties construeren, en dan nog in twee richtingen, dit met elk
bestaand domein.

\subsection{Expliciete vertrouwensrelaties}

Transitieve vertrouwensrelaties kunnen alleen bestaan tussen Windows Server
domeinen in hetzelfde forest, tenzij de diverse forests minimaal op Windows
Server 2003 functioneel niveau staan. In dat geval kun je manueel tussen de root
domeinen van de forests bi-directionele en transitieve forest trusts
configureren, waardoor je een gederatie of ream van forests krijgt. Elk koppel
domeinen in een dergelijke realm vertrouwt elkaar wederzijds. Bij meerdere
forests moet men forest trusts configureren tussen elk koppel forests.
Realm trusts, zijn een veralgemening van forest trusts, die vertrouwenspaden
leggen tussen Windows Server 2008 domeinen en willekeurige Kerberos v5 realms.
Een realm trust kan zowel bi-directioneel als enkelvoudig, en zowel transitief
als niet-transitief gedefinieerd worden.
Forest en realm vertrouwensrelaties zijn expliciete vertrouwensrelaties: trusts
die je zelf maakt, in tegenstelling tot de vertrouwensrelaties die automatisch
gemaakt worden tijdens de creatie van het domein.
Een verkorte vertrouwensrelatie is ook een expliciete vertrouwenrelatie en
maakt het mogelijk om een vertrouwenspad tussen 2 domeinen, in grote en complexe
trees, te verkorten. Dit moet het aanmeldingsproces versnellen.
Het laatste type expliciete vertrouwensrelatie is de externe vertrouwensrelatie.
Dit is een enkelvoudige vertrouwingsrelatie waarbij één domein een ander domein
vertrouwt. Deze zijn niet-transitief. Je kan geen externe relatie instellen
tussen AD domeinen in hetzelfde forest (heeft al automatische relaties). Dit kan
wel ingesteld worden tussen:
\begin{itemize}
	\item individuele domeinen in een ander forest.
	\item met NT 4 domeinen
\end{itemize}

\section{Op welke diverse manieren kunnen vertrouwensrelaties gecreëerd en
gecontroleerd worden? Bespreek ook de optionele configuratiemogelijkheden}

Enkel de expliciete vertrouwensrelaties moeten manueel geconfigureerd worden.
Dit kan op twee verschillende manieren geconfigureerd worden. Als je een
expliciete vertrouwenrelatie wil maken, moet je beschikken over de domeinnamen
en een gebruikersaccount met machtiging om vertrouwensrelaties in beide domeinen
te maken. Elke vertrouwensrelatie krijgt een wachtwoord teogewezen, dat bekend
moet zijn bij de beheerders van beide domeinen van de vertrouwensrelatie. Na het
opzetten van de vertrouwensrelatie wordt dit wachtword niet meer gebruikt.

\subsection{Active Directory Domains en Trust snap-in}

Deze snap-in is beschikbaar in domein.msc. Het aanmaken van de
vertrouwensrelatie kan door met de rechmuisknop op een domein te klikken en
achtereenvolgens Properties en de Trusts tabpagina te selecteren, en daarna de
New Trust wizard op te starten.

\subsection{Via de command-line}

In de command prompt kan je een vertrouwensrelatie configureren met de netdom
trust opdracht. Met netdom query trust krijg je een overzicht van alle
vertrouwensrelaties en hun toestand.

\subsection{Optionele configuratiemogelijkheden}
\begin{itemize}
	\item Standaard worden alle gebruikers van het trusted domein opgenomen
		in de Authenticated Users impliciete groep van het trusting
		domein. Men kan echter ook voor selective authentication kiezen,
		waardoor dit per individuele gebruiker of gebruikersgroep
		expliciet moet ingesteld worden.
	\item Indien men gebruik maakt van SID Filtering, dan wordt enkel
		rekening gehouden met de SID opgeslagen in het objectSid
		attribuut van de objecten in het trusted domein. Indien men SID
		Filtering uitschakelt, dan verwerkt het trusting domein ook de
		SIDs opgeslagen in het sIDHistory attribuut. Malfide beheerders
		in het trusted domein kunnen langs deze weg zichzelf meer
		machtigingen en rechten toeeigenen in het trusting domein.
\end{itemize}

\section{Welke verschillen zijn er in praktijk tussen NT 4.0 en Windows Server
domeinstructuren? Bespreek de alternatieve mogelijkheden bij de conversie van
een NT 4.0 domeinstructuur naar een Windows Server omgeving.}

Een eerste verschil tussen NT4 en Windows Server is het conceptueel onderscheid
tussen master domeinen en resource domeinen. Een master domein bevat gebruikers
en groepen, terwijl een resource domein lidservers bevat die diensten aanbieden
aan de gebruikers, en zelf nauwelijks gebruikers bevat. De NT4 domeinstructuren
bestaan meestal uit één (of meerdere) master domeinen, en meerdere resource
domeinen. Er worden bidirectionele vertrouwensrelaties aangemaakt tussen alle
masterdomeinen onderling, en unidirectionele vertrouwensrelaties waarbij elk
resourcedomein elk masterdomein vertrouwt.

De omschakeling van NT4 naar Windows Server moet geleidelijk en gefaseerd
gebeuren. Het aantal domeinen moet hierbij dalen. De organizational units (OUs)
van AD vervangen het conceptuele onderscheid tussen resource- en masterdomeinen.
De upgrade begint steeds bovenaan in de domeinhiërarchie: het masterdomein
krijgt als eerste een upgrade, daarna volgen de resource domeinen.

Bestaande domeinen kunnen worden gesimuleerd door OUs in het Windows Server
Domein. We bekomen zo een getrouwe weerspiegeling van de oude structuur.
Eventueel kunnen aanvullende structuren worden toegevoegd om een meer
gedetaileerde ordening te verkrijgen. Op deze manier wordt het aantal domeinen
en trusts gereduceerd.

Indien er bepaalde bedrijfseenheden als afzonderlijke organisaties moeten worden
behandeld, is een forest met afzonderlijke trees een goede oplossing. De
gebruikersaccounts worden dan verplaatst naar de domeinen met de bronnen die ze
gebruiken, in plaats van het centrale root domein.

Indien de oude structuur meerdere NT4 master domeinen bevat, zijn er hiervoor
een aantal mogelijke oorzaken. In eerste instantie kan het zijn dat een enkel
master domein te veel gebruikers en groepen zou bevatten. Deze veroorzaken
immers instabiliteit van de SAM databank. Dit probleem is meteen opgelost door
AD, omdat het veel schaalbaarder is. Een andere mogelijk oorzaak is de
geografische situering. Het is mogelijk dat de verschillende geografische
locaties verbonden zijn door links met kleine bandbreedte. Dit probleem kan
worden opgevangen door AD sites te configureren. Het replicatie verkeer kan nog
verder worden beperkt door gebruik te maken van RODC's. Een derede mogelijkheid
is de noodzaak aan een versschillende wachwoordbeleid voor bepaalde
gebruikersgroepen. Dit kan worden opgevangen door het domein functioneel niveau
te verheffen naar Windows Server 2008, en een fine grained password policy in te
stellen. Als laatste oorzaak voor verschillende NT4 master domeinen kan het zijn
dat bepaalde onderdelen van de organizatie controle moeten kunnen uitoefenen
over eigen bronnen en gebruikers. Dit is bij de invoering van AD de enige reden
om aparte domeinen te behouden in de verschillende sites. De migratie gebeurt op
een vand de volgende manieren:
\begin{enumerate}
	\item Elk NT4 domein wordt geupgraded naar de root van een Windows
		Server tree in hetzelfde forest. Alle gemachtigde gebruikers
		krijgen hierdoor potentiële toegang tot alle bronnen in alle
		domeinen van het forest. Het forest kan een gemeenschappelijk
		schema, gemeeschppelijke configuratiegegevens en een
		gemeenschappelijke global catalog delen.
	\item Als alternatief kan elk NT4 domein woren geupgraded naar een
		subdomein van een artificieel root domein van dezelfde tree. Dit
		root domein heet een structural of placeholder domein omdat het
		resources noch accounts bevat. Het is echter wel een uistekende
		plaats om de global catalog onder te brengen.
\end{enumerate}

Als laatste mogelijkheid voor migratie kan er besloten worden om alle NT4
domeinen samen te voegen tot één groot Windows Server domein. Dit kan op twee
manieren. In de eerste manier worden alle domeinen eerst samgevoegd, en
vervolgens geupgraded naar Windows Server. De oorspronkelijke domeinen worden
dan omgezet in OUs. Er zijn hiervoor geen hulpmiddelen. De migratie gebeurt
volledig manueel met behulp van commando lijn opdrachten. Als alternatief worden
de afzonderlijkde domeinen eerst geupgraded naar Windows Server. Vervolgens
wordt in elk van de masterdomeinen voor de resourcedomeinen een OU aangemaakt.
De machines in de resourcedomeinen worden dan verplaatst naar de OUs in de
masterdomeinen.
